\documentclass{amsart}
\usepackage{mathpazo}
\begin{document}
\title{The Feeling Of Power}

\author{Isaac Asimov}
\date{1957}
\maketitle
Jehan Shuman was used to dealing with the men in authority on
long-embattled earth. He was only a civilian but he originated
programming patterns that resulted in self-directing war computers of
the highest sort. Generals, consequently listened to him. Heads of
congressional committees too.

There was one of each in the special lounge of New Pentagon. General
Weider was space-burned and had a small mouth puckered almost into a
cipher. He smoked Denebian tobacco with the air of one whose
patriotism was so notorious, he could be allowed such liberties.

Shuman, tall, distinguished, and Programmer-first-class, faced them
fearlessly.

He said, "This, gentlemen, is Myron Aub."

"The one with the unusual gift that you discovered quite by accident,"
said Congressman Brant placidly. "Ah." He inspected the little man
with the egg-bald head with amiable curiosity.

The little man, in return, twisted the fingers of his hands anxiously.
He had never been near such great men before. He was only an aging
low-grade technician who had long ago failed all tests designed to
smoke out the gifted ones among mankind and had settled into the rut
of unskilled labor. There was just this hobby of his that the great
Programmer had found out about and was now making such a frightening
fuss over.

General Weider said, "I find this atmosphere of mystery childish."

"You won't in a moment," said Shuman. "This is not something we can
leak to the firstcomer. Aub!" There was something imperative about his
manner of biting off that one-syllable name, but then he was a great
Programmer speaking to a mere technician. "Aub! How much is nine times
seven?"

Aub hesitated a moment. His pale eyes glimmered with a feeble anxiety.

"Sixty-three," he said.

Congressman Brant lifted his eyebrows. "Is that right?"

"Check it for yourself, Congressman."

The congressman took out his pocket computer, nudged the milled edges
twice, looked at its face as it lay there in the palm of his hand, and
put it back. He said, "Is this the gift you brought us here to
demonstrate. An illusionist?"

"More than that, sir. Aub has memorized a few operations and with them
he computes on paper."

"A paper computer?" said the general. He looked pained.

"No, sir," said Shuman patiently. "Not a paper computer. Simply a
piece of paper. General, would you be so kind as to suggest a number?"

"Seventeen," said the general.

"And you, Congressman?"

"Twenty-three."

"Good! Aub, multiply those numbers, and please show the gentlemen your
manner of doing it."

"Yes, Programmer," said Aub, ducking his head. He fished a small pad
out of one shirt pocket and an artist's hairline stylus out of the
other. His forehead corrugated as he made painstaking marks on the
paper.

General Weider interrupted him sharply. "Let's see that."

Aub passed him the paper, and Weider said, "Well, it looks like the
figure seventeen."

Congressman Brant nodded and said, "So it does, but I suppose anyone
can copy figures off a computer. I think I could make a passable
seventeen myself, even without practice."

"If you will let Aub continue, gentlemen," said Shuman without heat.

Aub continued, his hand trembling a little. Finally he said in a low
voice, "The answer is three hundred and ninety-one."

Congressman Brant took out his computer a second time and flicked it.
"By Godfrey, so it is. How did he guess?"

"No guess, Congressman," said Shuman. "He computed that result. He did
it on this sheet of paper."

"Humbug," said the general impatiently. "A computer is one thing and
marks on a paper are another."

"Explain, Aub," said Shuman.

"Yes, Programmer. Well, gentlemen, I write down seventeen, and just
underneath it I write twenty-three. Next I say to myself: seven times
three -"

The congressman interrupted smoothly, "Now, Aub, the problem is
seventeen times twenty-three."

"Yes, I know," said the little technician earnestly, "but I start by
saying seven times three because that's the way it works. Now seven
times three is twenty-one."

"And how do you know that?" asked the congressman.

"I just remember it. It's always twenty-one on the computer. I've
checked it any number of times."

"That doesn't mean it always will be, though, does it?" said the
congressman.

"Maybe not," stammered Aub. "I'm not a mathematician. But I always get
the right answers, you see."

"Go on."

"Seven times three is twenty-one, so I write down twenty-one. Then one
times three is three, so I write down three under the two of
twenty-one."

"Why under the two?" asked Congressman Brant at once.

"Because - " Aub looked helplessly at his superior for support. "It's
difficult to explain."

Shuman said, "If you will accept his work for the moment, we can leave
the details for the mathematicians."

Brant subsided.

Aub said, "Three plus two makes five, you see, so the twenty- one
becomes a fifty-one. Now you let that go for a while and start fresh.
You multiply seven and two, that's fourteen, and one and two, that's
two. Put them down like this and it adds up to thirty-four. Now if you
put the thirty-four under the fifty-one this way and add them, you get
three hundred and ninety-one, and that's the answer."

There was an instant's silence and then General Weider said, "I don't
believe it. He goes through this rigmarole and makes up numbers and
multiplies and adds them this way and that, but I don't believe it.
It's too complicated to be anything but horn-swoggling."

"Oh no, sir," said Aub in a sweat. "It only seems complicated because
you're not used to it. Actually the rules are quite simple and will
work for any numbers."

"Any numbers, eh?" said the general. "Come, then." He took out his own
computer (a severely styled GI model) and struck it at random. "Make a
five seven three eight on the paper. That's five thousand seven
hundred and thirty-eight."

"Yes, sir," said Aub, taking a new sheet of paper.

"Now" - more punching of his computer - "seven two three nine. Seven
thousand two hundred and thirty-nine."

"Yes, sir."

"And now multiply those two."

"It will take some time," quavered Aub.

"Take the time," said the general.

"Go ahead, Aub," said Shuman crisply.

Aub set to work, bending low. He took another sheet of paper and
another. The general took out his watch finally and stared at it. "Are
you through with your magic-making, Technician?"

"I'm almost done, sir. Here it is, sir. Forty-one million, five
hundred and thirty-seven thousand, three hundred and eighty-two." He
showed the scrawled figures of the result.

General Weider smiled bitterly. He pushed the multiplication contact
on his computer and let the numbers whirl to a halt. And then he
stared and said in a surprised squeak, "Great Galaxy, the fella's
right."

The President of the Terrestrial Federation had grown haggard in
office and, in private, he allowed a look of settled melancholy to
appear on his sensitive features. The Denebian War, after its early
start of vast movement and great popularity, had trickled down into a
sordid matter of maneuver and counter-maneuver, with discontent rising
steadily on earth. Possibly, it was rising on Deneb, too.

And now Congressman Brant, head of the important Committee on Military
Appropriations, was cheerfully and smoothly spending his half-hour
appointment spouting nonsense.

"Computing without a computer," said the president impatiently, "is a
contradiction in terms."

"Computing," said the congressman, "is only a system for handling
data. A machine might do it, or the human brain might. Let me give you
an example." And, using the new skills he had learned, he worked out
sums and products until the president, despite himself, grew
interested.

"Does this always work?"

"Every time, Mr. President. It is foolproof."

"Is it hard to learn?"

"It took me a week to get the real hang of it. I think you would do
better."

"Well," said the president, considering, "it's an interesting parlor
game, but what is the use of it?"

"What is the use of a newborn baby, Mr. President? At the moment there
is no use, but don't you see that this points the way toward
liberation from the machine. Consider, Mr. President" - the
congressman rose and his deep voice automatically took on some of the
cadences he used in public debate - "that the Denebian War is a war of
computer against computer. Their computers forge an impenetrable
shield of countermissiles against our missiles, and ours forge one
against theirs. If we advance the efficiency of our computers, so do
they theirs, and for five years a precarious and profitless balance
has existed.

"Now we have in our hands a method for going beyond the computed,
leapfrogging it, passing through it. We will combine the mechanics of
computation with human thought; we will have the equivalent of
intelligent computers, billions of them. I can't predict what the
consequences will be in detail, but they will be incalculable. And if
Deneb beats us to the punch, they may be unimaginably catastrophic."

The president said, troubled, "What would you have me do?"

"Put the power of the administration behind the establishment of a
secret project on human computation. Call it Project Number, if you
like. I can vouch for my committee, but I will need the administration
behind me."

"But how far can human computation go?"

"There is no limit. According to Programmer Shuman, who first
introduced me to this discovery - "

"I've heard of Shuman, of course."

"Yes. Well, Dr. Shuman tells me that in theory there is nothing the
computer can do that the human mind cannot do. The computer merely
takes a finite amount of data and performs a finite amount of
operations on them. The human mind can duplicate the process."

The president considered that. He said, "If Shuman says this, I am
inclined to believe him - in theory. But, in practice, how can anyone
know how a computer works?"

Brant laughed genially. "Well, Mr. President, I asked the same
question. It seems that at one time computers were designed directly
by human beings. Those were simple computers, of course, this being
before the time of the rational use of computers to design more
advanced computers had been established."

"Yes, yes. Go on."

"Technician Aub apparently had, as his hobby, the reconstruction of
some of these ancient devices, and in so doing he studied the details
of their workings and found he could imitate them. The multiplication
I just performed for you is an imitation of the workings of a
computer."

"Amazing!"

The congressman coughed gently. "If I may make another point, Mr.
President - the further we can develop this thing, the more we can
divert our federal effort from computer production and computer
maintenance. As the human brain takes over, more of our energy can be
directed into peacetime pursuits and the impingement of war on the
ordinary man will be less. This will be most advantageous for the
party in power, of course."

"Ah," said the president, "I see your point. Well, sit down,
Congressman, sit down. I want some time to think about this. But
meanwhile, show me that multiplication trick again. Let's see if I
can't catch the point of it."

Programmer Shuman did not try to hurry matters. Loesser was
conservative, very conservative, and liked to deal with computers as
his father and grandfather had. Still, he controlled the West European
computer combine, and if he could be persuaded to join Project Number
in full enthusiasm, a great deal would be accomplished.

But Loesser was holding back. He said, "I'm not sure I like the idea
of relaxing our hold on computers. The human mind is a capricious
thing. The computer will give the same answer to the same problem each
time. What guarantee have we that the human mind will do the same?"

"The human mind, Computer Loesser, only manipulates facts. It doesn't
matter whether the human mind or a machine does it. They are just
tools."

"Yes, yes. I've gone over your ingenious demonstration that the mind
can duplicate the computer, but it seems to me a little in the air.
I'll grant the theory, but what reason have we for thinking that
theory can be converted to practice?"

"I think we have reason, sir. After all, computers have not always
existed. The cavemen with their triremes, stone axes, and railroads
had no computers."

"And possibly they did not compute."

"You know better than that. Even the building of a railroad or a
ziggurat called for some computing, and that must have been without
computers as we know them."

"Do you suggest they computed in the fashion you demonstrate?"

"Probably not. After all, this method - we call it 'graphitics,' by
the way, from the old European word 'grapho,' meaning 'to write' - is
developed from the computers themselves, so it cannot have antedated
them. Still, the cave men must have had some method, eh?"

"Lost arts! If you're going to talk about lost arts - "

"No, no. I'm not a lost art enthusiast, though I don't say there may
not be some. After all, man was eating grain before hydroponics, and
if the primitives ate grain, they must have grown it in soil. What
else could they have done?"

"I don't know, but I'll believe in soil growing when I see someone
grow grain in soil. And I'll believe in making fire by rubbing two
pieces of flint together when I see that too."

Shuman grew placative. "Well, let's stick to graphitics. It's just
part of the process of etherealization. Transportation by means of
bulky contrivances is giving way to mass transference. Communications
devices become less massive and more efficient constantly. For that
matter, compare your pocket computer with the massive jobs of a
thousand years ago. Why not, then, the last step of doing away with
computers altogether? Come, sir, Project Number is a going concern;
progress is already headlong. But we want your help. If patriotism
doesn't move you, consider the intellectual adventure involved."

Loesser said skeptically, "What progress? What can you do beyond
multiplication? Can you integrate a transcendental function?"

"In time, sir. In time. In the last month, I have learned to handle
division. I can determine, and correctly, integral quotients and
decimal quotients."

"Decimal quotients? To how many places?"

Programmer Shuman tried to keep his tone casual. "Any number!"

Loesser's jaw dropped. "Without a computer?"

"Set me a problem."

"Divide twenty-seven by thirteen. Take it to six places."

Five minutes later Shuman said, "Two point oh seven six nine two
three."

Loesser checked it. "Well, now, that's amazing. Multiplication didn't
impress me too much because it involved integers, after all, and I
thought trick manipulation might do it. But decimals - "

"And that is not all. There is a new development that is, so far, top
secret and which, strictly speaking, I ought not to mention. Still -
we may have made a break-through on the square root front."

"Square roots?"

"It involves some tricky points and we haven't licked the bugs yet,
but Technician Aub, the man who invented the science and who has
amazing intuition in connection with it, maintains he has the problem
almost solved. And he is only a technician. A man like yourself, a
trained and talented mathematician, ought to have no difficulty."

"Square roots," muttered Loesser, attracted.

"Cube roots, too. Are you with us?"

Loesser's hand thrust out suddenly. "Count me in."

General Weider stumped his way back and forth at the head of the room
and addressed his listeners after the fashion of a savage teacher
facing a group of recalcitrant students. It made no difference to the
general that they were the civilian scientists heading Project Number.
The general was over-all head, and he so considered himself at every
waking moment.

He said, "Now square roots are fine. I can't do them myself and I
don't understand the methods, but they're fine. Still, the project
will not be sidetracked into what some of you call the fundamentals.
You can play with graphitics any way you want to after the war is
over, but right now we have specific and very practical problems to
solve."

In a far corner Technician Aub listened with painful attention. He was
no longer a technician, of course, having been relieved of his duties
and assigned to the project, with a fine-sounding title and good pay.
But, of course, the social distinction remained, and the highly placed
scientific leaders could never bring themselves to admit him to their
ranks on a footing of equality. Nor, to do Aub justice, did he,
himself, wish it. He was as uncomfortable with them as they with him.

The general was saying, "Our goal is a simple one, gentlemen - the
replacement of the computer. A ship that can navigate space without a
computer on board can be constructed in one fifth the time and at one
tenth the expense of a computer-laden ship. We could build fleets five
times, ten times, as great as Deneb could if we could but eliminate
the computer.

"And I see something even beyond this. It may be fantastic now, a mere
dream, but in the future I see the manned missile!"

There was an instant murmur from the audience.

The general drove on. "At the present time our chief bottleneck is the
fact that missiles are limited in intelligence. The computer
controlling them can only be so large, and for that reason they can
meet the changing nature of anti-missile defenses in an unsatisfactory
way. Few missiles, if any, accomplish their goal, and missile warfare
is coming to a dead end, for the enemy, fortunately, as well as for
ourselves.

"On the other hand, a missile with a man or two within, controlling
flight by graphitics, would be lighter, more mobile, more intelligent.
It would give us a lead that might well mean the margin of victory.
Besides which, gentlemen, the exigencies of war compel us to remember
one thing. A man is much more dispensable than a computer. Manned
missiles could be launched in numbers and under circumstances that no
good general would care to undertake as far as computer-directed
missiles are concerned . . ."

He said much more, but Technician Aub did not wait.

Technician Aub, in the privacy of his quarters, labored long over the
note he was leaving behind. It read finally as follows:

"When I began the study of what is now called graphitics, it was no
more than a hobby. I saw no more in it than an interesting amusement,
an exercise of mind.

"When Project Number began, I thought that others were wiser than I,
that graphitics might be put to practical use as a benefit to mankind,
to aid in the production of really practical mass-transference devices
perhaps. But now I see it is to be used only for death and
destruction.

"I cannot face the responsibility involved in having invented
graphitics."

He then deliberately turned the focus of a protein depolarizer on
himself and fell instantly and painlessly dead.

They stood over the grave of the little technician while tribute was
paid to the greatness of his discovery.

Programmer Shuman bowed his head along with the rest of them but
remained unmoved. The technician had done his share and was no longer
needed, after all. He might have started graphitics, but now that it
had started, it would carry on by itself overwhelmingly, triumphantly,
until manned missiles were possible with who knew what else.

Nine times seven, thought Shuman with deep satisfaction, is
sixty-three, and I don't need a computer to tell me so. The computer
is in my own head.

And it was amazing the feeling of power that gave him.

\end{document}
