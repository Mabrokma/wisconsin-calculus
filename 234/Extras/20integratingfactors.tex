\section{Integrating a first order differential equation}
In math 222 we saw two methods for solving first order differential
equations, namely \textit{separation of variables} (for $y'=f(y)g(x)$)
and \textit{finding an integrating factor} (for linear equations).
Both methods can be seen as special cases of a more general method
that applies to more equations.  To explain this you need to know the
chain rule for functions of two variables, and also Clairaut's
theorem.

Suppose you have a differential equation
\begin{equation}\label{eq:01differential-equation}
  B(x, y) \frac{d y}{d x} + A(x, y)= 0
\end{equation}
and suppose you know a function $F(x, y)$ for which
\begin{equation}\label{eq:01integral-conditions}
  \pdd Fx  = A(x, y)\text{ and }  \pdd Fy  = B(x, y) 
\end{equation}
hold.  Then any solution $y(x)$ of the differential equation
\eqref{eq:01differential-equation} satisfies
\[
\frac{dF(x, y(x))}{dx} = \pdd Fx  \frac{dx}{dx} + \frac{\pd
F}{\pd y}\frac{dy}{dx} = A(x, y) + B(x, y) \frac{dy}{dx} =0,
\]
and therefore every solution of the differential equation
\eqref{eq:01differential-equation} also satisfies
\begin{equation}
  F(x, y(x)) = C  \text{ (i.e.\ $F(x, y)$ is constant.)}
  \label{eq:01diffeq-integrated}
\end{equation}
This solves the differential equation.  Our solution is given as an
implicit function.  Depending on how complicated the function $F$ is
you can now solve \eqref{eq:01diffeq-integrated} for $y(x)$ and get an
explicit representation of the solution.

To solve the differential equation \eqref{eq:01differential-equation} we therefore
have to find a function $F(x, y)$ of two variables which satisfies the
conditions \eqref{eq:01integral-conditions}.  By Clairaut's theorem we know that
we can do this provided $\pdd Ay  = \pdd Bx $ holds.  If
this condition is met then we know how to find $F$, so that's good news, but on
the other hand, if $A$ and $B$ are just arbitrary functions then it would be
surprising if they satisfy Clairaut's condition.  It looks like this method will
almost never work.  However, we are allowed to change the differential equation
a bit: we can multiply it (on both sides) with any nonzero function $m(x, y)$ to
get a new, equivalent diffeq:
\begin{equation}
  m(x,y)B(x, y) \frac{dy}{dx} + m(x, y)A(x, y) = 0.
  \label{eq:01differential-equation-times-m}
\end{equation}
This allows us to replace $A$ and $B$ by $\tilde A = m A$ and $\tilde B = mB$.  
Instead of looking for an $F$ which satisfies \eqref{eq:01integral-conditions}
we look for an $F$ which satisfies
\begin{equation}
  \pdd Fx  = m(x,y)A(x, y)\text{ and }  \pdd Fy  =
  m(x,y)B(x, y) .
  \label{eq:01integral-conditions-times-m}
\end{equation}
Clairaut says that such an $F$ can be found if 
\[
\frac{\pd mA}{\pd y} = \frac{\pd mB}{\pd x}
\]
holds.  This is a differential equation for $m$.  Usually it is not easier to
solve this equation for $m$, but sometimes it is.  Let's see how all this plays
out in the two examples we know (or once knew).

\subsection{Separable equations}
A separable equation is of the form $y' = f(x) g(y)$ and so we can write it as
\[
\frac{dy}{dx} - f(x) g(y) = 0.
\]
The integrating factor here is $m(x, y) = \frac{1}{g(y)}$, for if you multiply
with this $m$
you get
\[
\frac{1}{g(y)}\frac{dy}{dx} - f(x) = 0.
\]
An integral of the equation is then a function $F(x, y)$ which satisfies
\[
\pdd Fx  = -f(x) \text { and }
\pdd Fy  = \frac{1}{g(y)}.
\]
Clairaut's conditions are met, so such a function $F$ exists.  It is given by
\[
F(x, y) = \int \frac{dy}{g(y)}  - \int f(x) dx.
\]
\subsection{Linear First Order Equations}  The other kind of equation we learned
to solve in math 222 is $y' + P(x) = Q(x)$, which we can write as
\[
\frac{dy}{dx} + P(x)y-Q(x)  = 0.
\]
What kind of integrating factor shall we try?  From math 222 we know that an
integrating factor of the form $m(x)$ works, so let's multiply the equation with
a function of $x$ only.  We get
\[
\underbrace{m(x)}_{=B(x, y)}\frac{dy}{dx} +
\underbrace{m(x)\bigl(P(x)y-Q(x)\bigr)}_{ = A(x,y)}  = 0.
\]
we now want a function $F(x, y) $ with
\[
\pdd Fx  = m(x)\bigl(P(x)y-Q(x)\bigr)
\text{ and }
\pdd Fy  = m(x).
\]
Clairaut tells us that such an $F$ exists if 
\[
\frac{\pd\bigl[m(x)(P(x)y-Q(x))\bigr]}{\pd y}
=
\frac{\pd\bigl[m(x)\bigr]}{\pd x},
\]
i.e.
\[
m'(x) = m(x) P(x).
\]
This is the familiar diffeq we got in math 222 for the integrating factor
$m(x)$.  Solve this diffeq and you can integrate the equation.


