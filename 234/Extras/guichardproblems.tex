\chapter{From Guichard's Problem Collection}

\section{Partial Differentiation} %{{{1
\begin{multicols}{2}



Determine the equations and shapes of the cross-sections when
$x=0$, $y=0$, $x=y$, and describe the level curves.
Use a three-dimensional graphing tool to graph the surface.
\problem Let $f(x,y)=(x-y)^2$.  %{{{2
\answer
$z=y^2$, $z=x^2$, $z=0$, lines of slope 1
\endanswer

\problem Let $f(x,y)=|x|+|y|$.  %{{{2
\answer
$z=|y|$, $z=|x|$, $z=2|x|$, diamonds
\endanswer

\problem Let $f(x,y)=e^{-(x^2+y^2)}\sin(x^2+y^2)$.  %{{{2
\answer
$z=e^{-y^2}\sin(y^2)$, $z=e^{-x^2}\sin(x^2)$, 
$z=e^{-2x^2}\sin(2x^2)$, circles
\endanswer

\problem Let $f(x,y)=\sin(x-y)$.  %{{{2
\answer
$z=-\sin(y)$, $z=\sin(x)$, 
$z=0$, lines of slope 1
\endanswer

\problem Let $f(x,y)=(x^2-y^2)^2$.  %{{{2
\answer
$z=y^4$, $z=x^4$, 
$z=0$, hyperbolas
\endanswer







Determine whether each limit exists. If it does, find the limit
and prove that it is the limit; if it does not, explain how you know.

\problem $\DS\lim_{(x,y)\to(0,0)}{x^2\over x^2+y^2}$ %{{{2
\answer
No limit; use $x=0$ and $y=0$.
\endanswer

\problem $\DS\lim_{(x,y)\to(0,0)}{xy\over x^2+y^2}$ %{{{2
\answer
No limit; use $x=0$ and $x=y$.
\endanswer

\problem $\DS\lim_{(x,y)\to(0,0)}{xy\over 2x^2+y^2}$ %{{{2
\answer
No limit; use $x=0$ and $x=y$.
\endanswer

\problem $\DS\lim_{(x,y)\to(0,0)}{x^4-y^4\over x^2+y^2}$ %{{{2
\answer
Limit is zero.
\endanswer

\problem $\DS\lim_{(x,y)\to(0,0)}{\sin(x^2+y^2)\over x^2+y^2}$ %{{{2
\answer
Limit is 1.
\endanswer

\problem $\DS\lim_{(x,y)\to(0,0)}{xy\over \sqrt{2x^2+y^2}}$ %{{{2
\answer
Limit is zero.
\endanswer

\problem $\DS\lim_{(x,y)\to(0,0)} {e^{-x^2-y^2}-1\over x^2+y^2}$ %{{{2

\problem $\DS\lim_{(x,y)\to(0,0)}{x^3+y^3\over x^2+y^2}$ %{{{2

\problem Does the function $\DS f(x,y)={x-y\over 1+x+y}$  %{{{2
have any discontinuities?  What about 
$\DS f(x,y)={x-y\over 1+x^2+y^2}$?  Explain.





\problem Find $f_x$ and $f_y$ where $\DS f(x,y)=\cos(x^2y)+y^3$. %{{{2
\answer
$-2xy\sin(x^2y)$, $-x^2\sin(x^2y)+3y^2$
\endanswer

\problem Find $f_x$ and $f_y$ where $\DS f(x,y)={xy\over x^2+y}$. %{{{2
\answer
$(y^2-x^2y)/(x^2+y)^2$, $x^3/(x^2+y)^2$
\endanswer

\problem Find $f_x$ and $f_y$ where $\DS f(x,y)=e^{x^2+y^2}$. %{{{2
\answer
$2xe^{x^2+y^2}$, $2ye^{x^2+y^2}$
\endanswer

\problem Find $f_x$ and $f_y$ where $\DS f(x,y)=xy\ln(xy)$. %{{{2
\answer
$y\ln(xy)+y$, $x\ln(xy)+x$
\endanswer

\problem Find $f_x$ and $f_y$ where $\DS f(x,y)=\sqrt{1-x^2-y^2}$. %{{{2
\answer
$-x/\sqrt{1-x^2-y^2}$, $-y/\sqrt{1-x^2-y^2}$
\endanswer

\problem Find $f_x$ and $f_y$ where $\DS f(x,y)=x\tan(y)$. %{{{2
\answer
$\tan y$, $x\sec^2 y$
\endanswer

\problem Find $f_x$ and $f_y$ where $\DS f(x,y)={1\over xy}$. %{{{2
\answer
$-1/(x^2y)$, $-1/(xy^2)$
\endanswer

\problem Find an equation for the plane tangent to  %{{{2
$\DS 2x^2+3y^2-z^2=4$ at
$(1,1,-1)$. 
\answer
$z=-2(x-1)-3(y-1)-1$
\endanswer

\problem Find an equation for the plane tangent to  %{{{2
$\DS f(x,y)=\sin(xy)$ at
$(\pi,1/2,1)$. 
\answer
$z=1$
\endanswer

\problem Find an equation for the plane tangent to  %{{{2
$\DS f(x,y)=x^2+y^3$ at
$(3,1,10)$. 
\answer
$z=6(x-3)+3(y-1)+10$
\endanswer

\problem Find an equation for the plane tangent to  %{{{2
$\DS f(x,y)=x\ln(xy)$ at
$(2,1/2,0)$. 
\label{ex:ln tan plane}
\answer
$z=(x-2)+4(y-1/2)$
\endanswer

\problem Find an equation for the line normal to  %{{{2
$\DS x^2+4y^2=2z$ at
$(2,1,4)$. 
\answer
$\vr (t)=\langle 2,1,4\rangle+t\langle 2,4,-1\rangle$
\endanswer


\problem Explain in your own words why, when taking a partial derivative %{{{2
of a function of multiple variables, we can treat the variables not
being differentiated as constants.

\problem Consider a differentiable function, $f(x,y)$.  Give physical %{{{2
interpretations of the meanings of $f_x(a,b)$ and $f_y(a,b)$ as they
relate to the graph of $f$.

\problem In much the same way that we used the tangent line to %{{{2
approximate the value of a function from single variable calculus,
we can use the tangent plane to approximate a function from
multivariable calculus.  Consider the tangent plane found in
Exercise~\ref{ex:ln tan plane}. Use this plane to approximate
$f(1.98, .4)$.

\problem Suppose that one of your colleagues has calculated the partial %{{{2
derivatives of a given function, and reported to you that
$f_x(x,y)=2x+3y$ and that $f_y(x,y)=4x+6y$.  Do you believe them?
Why or why not?  If not, what answer might you have accepted for
$f_y$?

\problem Suppose $f(t)$ and $g(t)$ are single variable differentiable %{{{2
functions.  Find $\partial z/\partial x$ and
$\partial z/\partial y$ for each of the following two variable functions.

\begin{itemize}
\item[a.] $z=f(x)g(y)$
\item[b.] $z=f(xy)$
\item[c.] $z=f(x/y)$
\end{itemize}






\problem Use the chain rule to compute $dz/dt$ for %{{{2
$z=\sin(x^2+y^2)$, $x=t^2+3$, $y=t^3$.
\answer
$4xt\cos(x^2+y^2)+6yt^2\cos(x^2+y^2)$
\endanswer

\problem Use the chain rule to compute $dz/dt$ for %{{{2
$z=x^2y$, $x=\sin(t)$, $y=t^2+1$.
\answer
$2xy\cos t+2x^2t$
\endanswer

\problem Use the chain rule to compute $\partial z/\partial s$ and  %{{{2
$\partial z/\partial t$ for
$z=x^2y$, $x=\sin(st)$, $y=t^2+s^2$.
\answer
$2xyt\cos(st)+2x^2s$, $2xys\cos(st)+2x^2t$
\endanswer

\problem Use the chain rule to compute $\partial z/\partial s$ and  %{{{2
$\partial z/\partial t$ for
$z=x^2y^2$, $x=st$, $y=t^2-s^2$.
\answer
$2xy^2t-4yx^2s$, $2xy^2s+4yx^2t$
\endanswer

\problem Use the chain rule to compute $\partial z/\partial x$ and  %{{{2
$\partial z/\partial y$ for $2x^2+3y^2-2z^2=9$.
\answer
$x/z$, $3y/(2z)$
\endanswer

\problem Use the chain rule to compute $\partial z/\partial x$ and  %{{{2
$\partial z/\partial y$ for $2x^2+y^2+z^2=9$.
\answer
$-2x/z$, $-y/z$
\endanswer


\problem Find $\frac{dy}{dx}$ in each case both by Implicit %{{{2
Differentiation and by the Implicit Function Theorem.

\begin{enumerate}
\item $x^3+y^3+3xy=6$
\item $\sin(xy)=4$
\item $e^{x}+xy+\ln(y)=12$
\end{enumerate}

\problem Describe the points on the unit circle, $x^2+y^2=1$, where we %{{{2
{\em cannot\/} define one variable in terms of the other.

\problem Describe the points on the unit sphere, $x^2+y^2+z^2=1$, where %{{{2
we {\em cannot} define one of the variables in terms of the other two.





\problem Find $D_\vu f$ for $\DS f=x^2+xy+y^2$ in the direction of $\vu %{{{2
=\langle 2,1\rangle$ at the point $(1,1)$.  \answer $9\sqrt5/5$
\endanswer

\problem Find $D_\vu f$ for $\DS f=\sin(xy)$ in the direction of $\vu %{{{2
=\langle -1,1\rangle$ at the point $(3,1)$.  \answer $\sqrt2\cos3$
\endanswer

\problem Find $D_\vu f$ for $\DS f=e^x\cos(y)$ in the direction 30 degrees %{{{2
from the positive $x$ axis at the point $(1,\pi/4)$.  \answer
$e\sqrt2(\sqrt3-1)/4$
\endanswer

\problem The temperature of a thin plate in the $x$-$y$ plane is $\DS %{{{2
T=x^2+y^2$. How fast does temperature change at the point $(1,5)$ moving in
a direction 30 degrees from the positive $x$ axis?  \answer $\sqrt3+5$
\endanswer

\problem Suppose the density of a thin plate at $(x,y)$ is $\DS %{{{2
1/\sqrt{x^2+y^2+1}$. Find the rate of change of the density at $(2,1)$ in a
direction $\pi/3$ radians from the positive $x$ axis.  \answer
$-\sqrt6(2+\sqrt3)/72$
\endanswer

\problem Suppose the electric potential at $(x,y)$ is %{{{2
$\DS\ln\sqrt{x^2+y^2}$. Find the rate of change of the potential at $(3,4)$
toward the origin and also in a direction at a right angle to the direction
toward the origin.
\answer $-1/5$, $0$
\endanswer

\problem A plane perpendicular to the $x$-$y$ plane contains the point %{{{2
$(2,1,8)$ on the paraboloid $z=x^2+4y^2$. The cross-section of the
paraboloid created by this plane has slope 0 at this point. Find an
equation of the plane.  \answer $4(x-2)+8(y-1)=0$
\endanswer

\problem A plane perpendicular to the $x$-$y$ plane contains the point %{{{2
$(3,2,2)$ on the paraboloid $36z=4x^2+9y^2$. The cross-section of the
paraboloid created by this plane has slope 0 at this point.  Find an
equation of the plane.  \answer $2(x-3)+3(y-2)=0$
\endanswer

\problem Suppose the temperature at $(x,y,z)$ is given by $\DS %{{{2
T=xy+\sin(yz)$. In what direction should you go from the point $(1,1,1)$ to
decrease the temperature as quickly as possible? What is the rate of change
of temperature in this direction?  \answer $\langle
-1,-1-\cos1,-\cos1\rangle$, $-\sqrt{2+2\cos1+2\cos^21}$
\endanswer

\problem Suppose the temperature at $(x,y,z)$ is given by $\DS T=xyz$. In %{{{2
what direction can you go from the point $(1,1,1)$ to maintain the same
temperature?  \answer Any direction perpendicular to $\nab T=\langle
1,1,1\rangle$, for example, $\langle -1,1,0\rangle$
\endanswer

\problem Find an equation for the plane tangent to $\DS x^2-3y^2+z^2=7$ at %{{{2
$(1,1,3)$.  \answer $2(x-1)-6(y-1)+6(z-3)=0$
\endanswer

\problem Find an equation for the plane tangent to $\DS xyz=6$ at %{{{2
$(1,2,3)$.  \answer $6(x-1)+3(y-2)+2(z-3)=0$
\endanswer

\problem Find an equation for the line normal to $\DS x^2+2y^2+4z^2=26 $ at %{{{2
$(2,-3,-1)$.  \answer $\langle 2+4t,-3-12t,-1-8t\rangle$
\endanswer

\problem Find an equation for the line normal to $\DS x^2+y^2+9z^2=56$ at %{{{2
$(4,2,-2)$.  \answer $\langle 4+8t,2+4t,-2-36t\rangle$
\endanswer

\problem Find an equation for the line normal to $\DS x^2+5y^2-z^2=0$ at %{{{2
$(4,2,6)$.  \answer $\langle 4+8t,2+20t,6-12t\rangle$
\endanswer

\problem Find the directions in which the directional derivative of %{{{2
$f(x,y)=x^2+\sin(xy)$ at the point $(1,0)$ has the value 1.

\problem A bug is crawling on the surface of a hot plate, the temperature %{{{2
of which at the point $x$ units to the right of the lower left corner and
$y$ units up from the lower left corner is given by $T(x,y)=100-x^2-3y^3$.


\subprob If the bug is at the point $(2,1)$, in what direction should it
move to cool off the fastest?  How fast will the temperature drop in this
direction?

\subprob If the bug is at the point $(1,3)$, in what direction should it
move in order to maintain its temperature?


\problem Suppose that $g(x,y)=y-x^2$.  Find the gradient at the point $(-1, %{{{2
3)$.  Sketch the level curve to the graph of $g$ when $g(x,y)=2$, and plot
both the tangent line and the gradient vector at the point $(-1,3)$. (Make
your sketch large).  What do you notice, geometrically?

\problem The gradient $\nab(f)$ is a vector valued function of two %{{{2
variables.  Prove the following gradient rules.  Assume $f(x,y)$ and
$g(x,y)$ are differentiable functions.

\subprob $\nab(fg)=f\nab(g)+g\nab(f)$

\subprob $\nab(f/g)=(g\nab f - f \nab g)/g^2$

\subprob $\nab((f(x,y))^n)=nf(x,y)^{n-1}\nab f$





\problem Let $\DS f=xy/(x^2+y^2)$; compute $f_{xx}$, $f_{yx}$, and %{{{2
$f_{yy}$.  \answer $f_{xx}=(2x^3y-6xy^3)/(x^2+y^2)^3$,
$f_{yy}=(2xy^3-6x^3y)/(x^2+y^2)^3$
\endanswer

\problem Find all first and second partial derivatives of $x^3y^2+y^5$. %{{{2
\answer $f_x=3x^2y^2$, $f_y=2x^3y+5y^4$, $f_{xx}=6xy^2$,
$f_{yy}=2x^3+20y^3$, $f_{xy}=6x^2y$
\endanswer

\problem Find all first and second partial derivatives of $4x^3+xy^2+10$. %{{{2
\answer $f_x=12x^2+y^2$, $f_y=2xy$, $f_{xx}=24x$, $f_{yy}=2x$, $f_{xy}=2y$
\endanswer

\problem Find all first and second partial derivatives of $x\sin y$. %{{{2
\answer $f_x=\sin y$, $f_y=x\cos y$, $f_{xx}=0$, $f_{yy}=-x\sin y$,
$f_{xy}=\cos y$
\endanswer

\problem Find all first and second partial derivatives of %{{{2
$\sin(3x)\cos(2y)$.

\problem Find all first and second partial derivatives of $e^{x+y^2}$. %{{{2

\problem Find all first and second partial derivatives of %{{{2
$\ln\sqrt{x^3+y^4}$.

\problem Find all first and second partial derivatives of $z$ with respect %{{{2
to $x$ and $y$ if $x^2+4y^2+16z^2-64=0$.

\problem Find all first and second partial derivatives of $z$ with respect %{{{2
to $x$ and $y$ if $xy+yz+xz=1$.

\problem How many third-order derivatives does a function of 2 variables %{{{2
have?  How many of these are distinct?

\problem How many $n$th order derivatives does a function of 2 variables %{{{2
have?  How many of these are distinct?

\problem Let $\alpha$ and $k$ be constants.  Prove that the function %{{{2
$u(x,t)=e^{-\alpha^2k^2t}\sin(kx)$ is a solution to the heat equation
$u_t=\alpha^2u_{xx}$

\problem Let $a$ be a constant.  Prove that $u=\sin(x-at)+\ln(x+at)$ is a %{{{2
solution to the wave equation $u_{tt}=a^2u_{xx}$.

\problem Let $f(x,y)$ be a continuous differentiable function.  Analyze the %{{{2
level curves near a critical value if that critical value is a max or a
min.  What if the level curve is a saddle point?





\problem Find all local maximum and minimum points of $f=x^2+4y^2-2x+8y-1$. %{{{2
\answer minimum at $(1,-1)$
\endanswer

\problem Find all local maximum and minimum points of $f=x^2-y^2+6x-10y+2$. %{{{2
\answer none
\endanswer

\problem Find all local maximum and minimum points of $f=xy$.  \answer none %{{{2
\endanswer

\problem Find all local maximum and minimum points of $f=9+4x-y-2x^2-3y^2$. %{{{2
\answer maximum at $(1,-1/6)$
\endanswer

\problem Find all local maximum and minimum points of $f=x^2+4xy+y^2-6y+1$. %{{{2
\answer none
\endanswer

\problem Find all local maximum and minimum points of %{{{2
$f=x^2-xy+2y^2-5x+6y-9$.  \answer minimum at $(2,-1)$
\endanswer

\problem A six-sided rectangular box is to hold $1/2$ cubic meter; what %{{{2
shape should the box be to minimize surface area?  \answer a cube $1/\root
3 \of {2}$ on a side
\endanswer

\problem The post office will accept packages whose combined length and %{{{2
girth is at most 130 inches (girth is the maximum distance around the
package perpendicular to the length). What is the largest volume that can
be sent in a rectangular box?  \answer $65/3\times 65/3\times 130/3$
\endanswer

\problem The bottom of a rectangular box costs twice as much per unit area %{{{2
as the sides and top. Find the shape for a given volume that will minimize
cost.  \answer It has a square base, and is one and one half times as tall
as wide.  If the volume is $V$ the dimensions are $\root 3 \of {2V/3}\times
\root 3 \of {2V/3}\times \root 3\of {9V/4}$.
\endanswer

\problem Using the methods of this section, find the shortest distance from %{{{2
the origin to the plane $x+y+z=10$.  \answer $\sqrt{100/3}$
\endanswer

\problem Using the methods of this section, find the shortest distance from %{{{2
the point $(x_0,y_0,z_0)$ to the plane $ax+by+cz=d$.  You may assume that
$c\not=0$; use of Maple or similar software is recommended.  \answer
$|ax_0+by_0+cz_0-d|/\sqrt{a^2+b^2+c^2}$
\endanswer

\problem A trough is to be formed by bending up two sides of a long metal %{{{2
rectangle so that the cross-section of the trough is an isosceles
trapezoid, as in figure~\ref{fig:trough}. If the width of the metal sheet
is 2 meters, how should it be bent to maximize the volume of the trough?
\answer The sides and bottom should all be $2/3$ meter, and the sides
should be bent up at angle $\pi/3$.
\endanswer

\problem Given the three points $(1,4)$, $(5,2)$, and $(3,-2)$, %{{{2
$\DS(x-1)^2+(y-4)^2+(x-5)^2+(y-2)^2+(x-3)^2+(y+2)^2$ is the sum of the
squares of the distances from point $(x,y)$ to the three points. Find $x$
and $y$ so that this quantity is minimized.  \answer $(3,4/3)$
\endanswer

\problem Suppose that $f(x,y)=x^2+y^2+kxy$. Find and classify the critical %{{{2
points, and discuss how they change when $k$ takes on different values.

\problem Find the shortest distance from the point $(0,b)$ to the parabola %{{{2
$y=x^2$.

\problem Find the shortest distance from the point $(0,0,b)$ to the %{{{2
paraboloid $z=x^2+y^2$.

\problem Consider the function $f(x,y)=x^3-3x^2y$. %{{{2

\subprob Show that $(0,0)$ is the only critical point of $f$.

\subprob Show that the discriminant test is inconclusive for $f$.

\subprob Determine the traces of $f$ obtained by setting $x=k$ for various
values of $k$.

\subprob What kind of critical point is $(0,0)$?


\problem Find the volume of the largest rectangular box with edges parallel %{{{2
to the axes that can be inscribed in the ellipsoid $2x^2+72y^2+18z^2=288$.





\problem A six-sided rectangular box is to hold $1/2$ cubic meter; what %{{{2
shape should the box be to minimize surface area?  \answer a cube
\endanswer

\problem The post office will accept packages whose combined length and %{{{2
girth are at most 130 inches (girth is the maximum distance around the
package perpendicular to the length). What is the largest volume that can
be sent in a rectangular box?  \answer $65/3\times 65/3\times 130/3$
\endanswer

\problem The bottom of a rectangular box costs twice as much per unit area %{{{2
as the sides and top. Find the shape for a given volume that will minimize
cost.  \answer It has a square base, and is one and one half times as tall
as wide.  If the volume is $V$ the dimensions are $\root 3 \of {2V/3}\times
\root 3 \of {2V/3}\times \root 3\of {9V/4}$.
\endanswer

\problem Using Lagrange multipliers, find the shortest distance from the %{{{2
point $(x_0,y_0,z_0)$ to the plane $ax+by+cz=d$.  \answer
$|ax_0+by_0+cz_0-d|/\sqrt{a^2+b^2+c^2}$
\endanswer

\problem Find all points on the surface $xy-z^2+1=0$ that are closest to %{{{2
the origin.  \answer $(0,0,1)$, $(0,0,-1)$
\endanswer

\problem The material for the bottom of an aquarium costs half as much as %{{{2
the high strength glass for the four sides. Find the shape of the cheapest
aquarium that hold a given volume $V$.  \answer $\root 3\of{4V}\times\root
3\of{4V}\times\root 3\of{V/16}$
\endanswer

\problem The plane $x-y+z=2$ intersects the cylinder $x^2+y^2=4$ in an %{{{2
ellipse. Find the points on the ellipse closest to and farthest from the
origin.  \answer Farthest: $(-\sqrt2,\sqrt2,2+2\sqrt2)$; closest:
$(2,0,0)$, $(0,-2,0)$
\endanswer

\end{multicols}
\section{Vector Functions} %{{{1

\begin{multicols}{2}


\problem Investigate the following curves %{{{2

\subprob $\vr =\tvek \sin t\\ \cos t\\ \cos 8t\ttor$.
\hfill
\subprob
$\vr =\tvek t\cos t\\ t\sin t\\ t\ttor$.

\subprob
$\vr =\tvek t\\ t^2\\ \cos t\ttor$.
\hfill
\subprob
$\vr =\tvek \cos(20t)\sqrt{1-t^2}\\ \sin(20t)\sqrt{1-t^2}\\ t\ttor$

\problem Find a vector function for the curve of intersection of %{{{2
$x^2+y^2=9$ and $y+z=2$.
\answer
$\tvek 3\cos t\\  3\sin t\\  2-3\sin t\ttor$
\endanswer

\problem A bug is crawling outward along the spoke of a wheel that lies along a %{{{2
radius of the wheel. The bug is crawling at 1 unit per second and the wheel is
rotating at 1 radian per second. Suppose the wheel lies in the $y$-$z$ plane
with center at the origin, and at time $t=0$ the spoke lies along the positive
$y$ axis. Find a vector function $\vr (t)$ for the position of the bug at
time $t$.
\answer
$\tvek 0,t\cos t,t\sin t\ttor$
\endanswer

\problem What is the difference between the parametric curves %{{{2
$\vu(t)=\tvek t\\ t\\ t^2 \ttor$, $\vv(t)=\tvek t^2\\ t^2\\ t^4 \ttor$, and
$\vw(t)=\tvek \sin(t)\\ \sin(t)\\ \sin^2(t) \ttor$ as $t$
runs over all real numbers?

\problem Plot each of the curves below in 2 dimensions, projected onto each of %{{{2
the three standard planes (the $x$-$y$, $x$-$z$, and $y$-$z$ planes).

\begin{itemize}
\item[a.] $\vv(t)=\tvek t\\  t^3\\  t^2 \ttor$,  $t$ ranges over all real numbers
\item[b.] $.f(t)=\tvek t^2\\  t-1\\  t^2+5 \ttor$  for $0\leq t \leq 3$
\end{itemize}

\problem Given points $A=(a_1, a_2, a_3)$ and $B=(b_1, b_2, b_3)$ give %{{{2
parametric equations for the line {\em segment} connecting $A$ and
$B$. Be sure to give appropriate $t$ values.

\problem With a parametric plot and a set of $t$ values, we can associate %{{{2
a `direction'.  For example, the curve $\tvek \cos t\\ \sin t
\ttor$ is the unit circle traced counterclockwise.  How can we amend
a set of given parametric equations and $t$ values to get the same
curve, only traced backwards?




\problem Find $\vr '$ and $\vT$ for %{{{2
$\vr  = \langle t^2,1,t\rangle$.
\answer
$\langle 2t,0,1\rangle$, $\vr '/\sqrt{1+4t^2}$
\endanswer

\problem Find $\vr '$ and $\vT$ for %{{{2
$\vr  = \langle \cos t, \sin 2t, t^2\rangle$.
\answer
$\langle -\sin t, 2\cos 2t,2t\rangle$,
$\vr '/\sqrt{\sin^2t + 4\cos^2(2t)+4t^2}$
\endanswer

\problem Find $\vr '$ and $\vT$ for %{{{2
$\vr  = \langle \cos(e^t),\sin(e^t),\sin t\rangle$.
\answer
$\langle -e^t\sin(e^t),e^t\cos(e^t),\cos t\rangle$,
$\vr '/\sqrt{e^{2t}+\cos^2 t}$
\endanswer

\problem Find a vector function for the line tangent to the helix %{{{2
$\langle \cos t,\sin t, t\rangle$ when $t=\pi/4$.
\answer
$\langle \sqrt2/2,\sqrt2/2,\pi/4\rangle+
t\langle -\sqrt2/2,\sqrt2/2,1\rangle$
\endanswer

\problem Find a vector function for the line tangent to  %{{{2
$\langle \cos t,\sin t, \cos 4t \rangle$ when $t=\pi/3$.
\answer
$\langle 1/2,\sqrt3/2,-1/2\rangle+
t\langle -\sqrt3/2,1/2,2\sqrt3\rangle$
\endanswer

\problem Find the cosine of the angle between the curves $\langle %{{{2
0,t^2,t\rangle$ and $\langle \cos(\pi t/2),\sin(\pi t/2), t\rangle$
where they intersect.
\answer
$2/\sqrt5/\sqrt{4+\pi^2}$
\endanswer

\problem Find the cosine of the angle between the curves $\langle %{{{2
\cos t,-\sin(t)/4,\sin t\rangle$ and $\langle \cos t,\sin t, \sin(2t)\rangle$
where they intersect.
\answer
$7\sqrt{5}\sqrt{17}/85$, $-9\sqrt{5}\sqrt{17}/85$
\endanswer

\problem Suppose that $|\vr (t)|=k$, for some constant $k$. This %{{{2
means that $\vr$ describes some path on the sphere of radius $k$
with center at the origin. Show that $\vr$ is perpendicular to ${\vr}'$ at every
point. Hint: Use Theorem~\ref{thm:vector derivative
properties}, part (d).
\label{exercise:derivative is perpendicular}

\problem A bug is crawling along the spoke of a wheel that lies along %{{{2
a radius of the wheel. The bug is crawling at 1 unit per second and
the wheel is rotating at 1 radian per second. Suppose the wheel lies
in the $y$-$z$ plane with center at the origin, and at time $t=0$ the
spoke lies along the positive $y$ axis and the bug is at the origin. 
Find a vector function $\vr (t)$
for the position of the bug at time $t$, the velocity vector
$\vr '(t)$, the unit tangent ${\vT}(t)$, and the speed of the bug
$|\vr '(t)|$.
\answer
$\langle 0,t\cos t,t\sin t\rangle$, 
$\langle 0,\cos t-t\sin t,\sin t+t\cos t\rangle$,
$\vr '/\sqrt{1+t^2}$, $\sqrt{1+t^2}$
\endanswer

\problem An object moves with velocity vector $\langle \cos t, \sin t, %{{{2
t\rangle$, starting at $\langle 0,0,0\rangle$ when $t=0$. Find the function
$\vr $ giving its location.
\answer
$\langle \sin t,1-\cos t,t^2/2\rangle$
\endanswer

\problem An object moves with velocity vector $\langle t, t^2, %{{{2
\cos t\rangle$, starting at $\langle 0,0,0\rangle$ when $t=0$.
Find the function
$\vr $ giving its location.
\answer
$\langle t^2/2,t^3/3,\sin t\rangle$
\endanswer

\problem What is the physical interpretation of the dot product of two %{{{2
vector valued functions?  What is the physical interpretation of the
cross product of two vector valued functions?

\problem Show, using the rules of cross products and differentiation, %{{{2
that
$${d\over dt} (\vr (t) \times \vr '(t))= 
\vr (t) \times \vr ''(t).$$

\problem Determine the point at which ${\vf}(t)=\langle t, t^2, t^3 %{{{2
\rangle$ and ${\vg}(t) =\langle \cos(t), \cos(2t), t+1 \rangle$
intersect, and find the angle between the curves at that point.  (Hint:
You'll need to set this one up like a line intersection problem,
writing one in $s$ and one in $t$.) If these two functions were the
trajectories of two airplanes on the same scale of time, would the
planes collide at their point of intersection?  Explain.




\problem Find the length of $\langle 3\cos t,2t,3\sin t\rangle$,  %{{{2
$t\in[0,2\pi]$.
\answer
$2\pi\sqrt{13}$
\endanswer

\problem Find the length of $\langle t^2,2,t^3\rangle$, $t\in[0,1]$. %{{{2
\answer
$(-8+13\sqrt{13})/27$
\endanswer

\problem Find the length of $\langle t^2,\sin t,\cos t\rangle$, $t\in[0,1]$. %{{{2
\answer
$\sqrt5/2+\ln(\sqrt5+2)/4$
\endanswer

\problem Find the length of the curve $y=x^{3/2}$, $x\in[1,9]$. %{{{2
\answer
$(85\sqrt{85}-13\sqrt{13})/27$
\endanswer

\problem Set up an integral to compute the length of %{{{2
$\langle \cos t, \sin t, e^t\rangle$, $t\in[0,5]$. (It is tedious but
not too difficult to compute this integral.)
\answer
$\int_0^5 \sqrt{1+e^{2t}}\,dt$
\endanswer

\problem Find the curvature of $\langle t,t^2,t\rangle$. %{{{2
\answer
$2\sqrt2/(2+4t^2)^{3/2}$
\endanswer

\problem Find the curvature of $\langle t,t^2,t^2\rangle$. %{{{2
\answer
$2\sqrt2/(1+8t^2)^{3/2}$
\endanswer

\problem Find the curvature of $\langle t,t^2,t^3\rangle$. %{{{2
\answer
$2\sqrt{1+9t^2+9t^4}/(1+4t^2+9t^4)^{3/2}$
\endanswer




\problem Let $\vr =\langle \cos t,\sin t,t\rangle$.  %{{{2
Compute $\vv $, ${\va}$,
$a_T$, and $a_N$.
\answer
$\langle -\sin t,\cos t,1\rangle$,
$\langle -\cos t, -\sin t,0\rangle$,
$0$, $1$
\endanswer

\problem Let $\vr =\langle \cos t,\sin t,t^2\rangle$.  %{{{2
Compute $\vv $, ${\va}$,
$a_T$, and $a_N$.
\answer
$\langle -\sin t,\cos t,2t\rangle$,
$\langle -\cos t, -\sin t,2\rangle$,
$4t/\sqrt{4t^2+1}$, $\sqrt{4t^2+5}/\sqrt{4t^2+1}$
\endanswer

\problem Let $\vr =\langle \cos t,\sin t,e^t\rangle$.  %{{{2
Compute $\vv $, ${\va}$,
$a_T$, and $a_N$.
\answer
$\langle -\sin t,\cos t,e^t\rangle$,
$\langle -\cos t, -\sin t,e^t\rangle$,
$e^{2t}/\sqrt{e^{2t}+1}$, $\sqrt{2e^{2t}+1}/\sqrt{e^{2t}+1}$
\endanswer

\problem Let $\vr =\langle e^t,\sin t,e^t\rangle$.  %{{{2
Compute $\vv $, $\va $,
$a_T$, and $a_N$.
\answer
$\langle e^t,\cos t,e^t\rangle$,
$\langle e^t, -\sin t,e^t\rangle$,
$(2e^{2t}-\cos t\sin t)/\sqrt{2e^{2t}+\cos^2 t}$, 
$\sqrt{2}e^t|\cos t+\sin t|/\sqrt{2e^{2t}+\cos^2 t}$
\endanswer

\problem Suppose an object moves so that its acceleration is given by %{{{2
$\va =\langle -3\cos t,-2\sin t,0\rangle$. At time $t=0$ the object
is at $(3,0,0)$ and its velocity vector is $\langle
0,2,0\rangle$. Find $\vv (t)$ and $\vr (t)$ for the object.
\answer
$\langle -3\sin t,2\cos t,0\rangle$,
$\langle 3\cos t, 2\sin t,0\rangle$
\endanswer

\problem Suppose an object moves so that its acceleration is given by %{{{2
$\va =\langle -3\cos t,-2\sin t,0\rangle$. At time $t=0$ the object
is at $(3,0,0)$ and its velocity vector is $\langle
0,2.1,0\rangle$. Find $\vv (t)$ and $\vr (t)$ for the object.
\answer
$\langle -3\sin t,2\cos t+0.1,0\rangle$,
$\langle 3\cos t, 2\sin t+t/10,0\rangle$
\endanswer

\problem Suppose an object moves so that its acceleration is given by %{{{2
$\va =\langle -3\cos t,-2\sin t,0\rangle$. At time $t=0$ the object
is at $(3,0,0)$ and its velocity vector is $\langle
0,2,1\rangle$. Find $\vv (t)$ and $\vr (t)$ for the object.
\answer
$\langle -3\sin t,2\cos t,1\rangle$,
$\langle 3\cos t, 2\sin t,t\rangle$
\endanswer

\problem Suppose an object moves so that its acceleration is given by %{{{2
$\va =\langle -3\cos t,-2\sin t,0\rangle$. At time $t=0$ the object
is at $(3,0,0)$ and its velocity vector is $\langle
0,2.1,1\rangle$. Find $\vv (t)$ and $\vr (t)$ for the object.
\answer
$\langle -3\sin t,2\cos t+1/10,1\rangle$,
$\langle 3\cos t, 2\sin t+t/10,t\rangle$
\endanswer


\end{multicols}

\def\tint#1{\int\int\int_{#1}}
\section{Multiple Integration} %{{{1

\begin{multicols}{2}


\problem Compute $\DS \int_{0}^{2}\int_{0}^{4} 1+x \,dy\,dx$. %{{{2
\answer
$16$
\endanswer

\problem Compute $\DS \int_{-1}^{1}\int_{0}^{2} x+y\,dy\,dx$. %{{{2
\answer
$4$
\endanswer


\problem Compute $\DS \int_{1}^{2}\int_{0}^{y} xy \,dx\,dy$. %{{{2
\answer
$15/8$
\endanswer


\problem Compute $\DS \int_{0}^{1}\int_{y^2/2}^{\sqrt y} \,dx\,dy$. %{{{2
\answer
$1/2$
\endanswer


\problem Compute $\DS \int_{1}^{2}\int_{1}^{x} {x^2\over y^2}\,dy\,dx$. %{{{2
\answer
$5/6$
\endanswer


\problem Compute $\DS \int_{0}^{1}\int_{0}^{x^2} {y\over e^x}\,dy\,dx$. %{{{2
\answer
$12-65/(2e)$.
\endanswer


\problem Compute $\DS \int_{0}^{\sqrt{\pi/2}}\int_{0}^{x^2} x\cos y\,dy\,dx$. %{{{2
\answer
$1/2$
\endanswer


\problem Compute $\DS \int_{0}^{\pi/2}\int_{0}^{\cos\theta}r^2 %{{{2
(\cos\theta-r) \,dr\,d\theta$.
\answer
$\pi/64$
\endanswer

\problem  Compute: $\DS \int_0^1\int_{\sqrt{y}}^1  %{{{2
\sqrt{x^3+1}\,dx\,dy$.
\answer
$(2/9)2^{3/2}-(2/9)$
\endanswer

\problem Compute: $\DS \int_0^1\int_{y^2}^1  %{{{2
y\sin(x^2)\,dx\,dy$.
\answer
$(1-\cos(1))/4$
\endanswer

\problem Compute: $\DS \int_0^1 \int_{x^2}^1 x\sqrt{1+y^2}\,dy\,dx$ %{{{2
\answer
$(2\sqrt2-1)/6$
\endanswer

\problem Compute: $\DS \int_0^1 \int_0^y %{{{2
{2\over\sqrt{1-x^2}}\,dx\,dy$
\answer
$\pi-2$
\endanswer

\problem Find the volume bounded by $z=x^2+y^2$ and $z=4$. %{{{2
\answer
$8\pi$
\endanswer


\problem Find the volume in the first octant %{{{2
bounded by $y^2=4-x$ and $y=2z$.
\answer
$2$
\endanswer


\problem Find the volume in the first octant %{{{2
bounded by $y^2=4x$, $2x+y=4$, $z=y$,
and $y=0$.
\answer
$5/3$
\endanswer

\problem Find the volume in the first octant %{{{2
bounded by $x+y+z=9$, $2x+3y=18$, and $x+3y=9$.
\answer
$81/2$
\endanswer

\problem Find the volume in the first octant %{{{2
bounded by $x^2+y^2=a^2$ and $z=x+y$.
\answer
$2a^3/3$
\endanswer

\problem Find the volume bounded by $4x^2+y^2=4z$ and $z=2$. %{{{2
\answer
$4\pi$
\endanswer

\problem Find the volume bounded by $z=x^2+y^2$ and $z=y$. %{{{2
\answer
$\pi/32$
\endanswer

\problem Find the average value of $f(x,y)=e^y\sqrt{x+e^y}$ on the %{{{2
rectangle with vertices $(0,0)$, $(4,0$), $(4,1)$ and $(0,1)$.

%% Balof
\problem Below is a weather map of Colorado.  Use %{{{2
the data to estimate the average temperature in the state using 4,
16 and 25 subdivisions.  Give both an upper and lower estimate.
Why do we like Colorado for this problem?  What
other state might we like?

\verb|\includegraphics{weathermap.ps}|

\problem Three cylinders of radius 1 intersect at right angles at the %{{{2
origin, as shown in figure~\ref{fig:three cylinders}. Find the
volume contained inside all three cylinders.
\answer
$16-8\sqrt{2}$
\endanswer

\begin{figure}
    \caption{Intersection of three cylinders.}
\end{figure}

\problem Prove that if $f(x,y)$ is integrable and if $\DS g(x,y)=\int_a^x %{{{2
\int_b^y f(s,t) \; dt \; ds$ then $g_{xy}=g_{yx}=f(x,y)$.

\problem Reverse the order of integration on each of the following integrals %{{{2

\begin{itemize}
\item[a.] $\DS\int_0^3 \int_0^{\sqrt{9-y}} f(x,y)\; dx \; dy$
\item[b.] $\DS\int_1^2 \int_0^{\ln x} f(x,y) \; dy \; dx $
\item[c.] $\DS\int_0^1 \int_{\arcsin y}^{\pi/2} f(x,y) \; dx \; dy$
\end{itemize}

%% Balof
\problem What are the parallels between Fubini's %{{{2
Theorem and Clairaut's Theorem?






\problem Find the volume above the $x$-$y$ plane, under the surface %{{{2
$r^2=2z$, and inside $r=2$.
\answer
$4\pi$
\endanswer

\problem Find the volume inside both $r=1$ and $r^2+z^2=4$. %{{{2
\answer
$32\pi/3-4\sqrt3\pi$
\endanswer

\problem Find the volume below $z=\sqrt{1-r^2}$ and above %{{{2
the top half of the cone $z=r$.
\answer
$(2-\sqrt2)\pi/3$
\endanswer

\problem Find the volume below  $z=r$, above the $x$-$y$ plane, and %{{{2
inside $r=\cos\theta$.
\answer
$4/9$
\endanswer

\problem Find the volume below  $z=r$, above the $x$-$y$ plane, and %{{{2
inside $r=1+\cos\theta$.
\answer
$5\pi/3$
\endanswer

\problem Find the volume between $x^2+y^2=z^2$ and $x^2+y^2=z$. %{{{2
\answer
$\pi/6$
\endanswer

\problem Find the area inside $r=1+\sin\theta$ and outside %{{{2
$r=2\sin\theta$. 
\answer
$\pi/2$
\endanswer

\problem Find the area inside both %{{{2
$r=2\sin\theta$ and $r=2\cos\theta$. 
\answer
$\pi/2-1$
\endanswer

\problem Find the area inside the four-leaf rose $r=\cos(2\theta)$ %{{{2
and outside $r=1/2$.
\answer
$\sqrt3/4+\pi/6$
\endanswer

\problem Investigate and describe the differences between the graphs of %{{{2
$r=\cos(2\theta)$ and $r=\sin(2\theta)$.

\problem Investigate and describe the differences between the graphs of %{{{2
$r=\cos(2k\theta)$ and $r=\cos((2k+1)\theta)$

\problem Figure \ref{fig:double flower} shows the plot of %{{{2
$r=1+4\sin(5\theta)$.

\begin{figure}
    \caption{$r=1+4\sin(5\theta)$}
\end{figure}

\begin{itemize}

\item[a.] Describe the behavior of the graph in terms of the given
    equation.  Specifically, explain maximum and minimum values, number
    of leaves, and the 'leaves within leaves'.

\item[b.] Give an integral or integrals to determine the area outside a
    smaller leaf but inside a larger leaf.


\item[c.] How would changing the value of $a$ in the equation
    $r=1+a\cos(5\theta)$ change the relative sizes of the inner and
    outer leaves? Focus on values $a\geq 1$.  (Hint: How would we change
    the maximum and minimum values?)

\end{itemize}

\problem Consider the integral $\DS\int\int_{D} {1\over\sqrt{x^2+y^2}} \; %{{{2
dA$, where $D$ is the unit disk centered at the origin.

\begin{itemize}

\item[a.] Why might this integral be considered improper?

\item[b.] Calculate the value of this integral over the annulus with outer
    radius 1 and inner radius $\delta$.

\item[c.] Obtain a value for the integral on the whole disk by letting
    $\delta$ approach 0.

\item[d.] For which values $\lambda$ can we replace the denominator with
    $(x^2+y^2)^\lambda$ in the original integral?

\end{itemize}






\problem Find the center of mass of a two-dimensional plate  %{{{2
that occupies the square $[0,1]\times[0,1]$
and has density
function $xy$.
\answer
$\bar x=\bar y=2/3$
\endanswer

\problem Find the center of mass of a two-dimensional plate  %{{{2
that occupies the triangle $0\le x\le1$, $0\le y\le x$,
and has density
function $xy$.
\answer
$\bar x=4/5$, $\bar y=8/15$
\endanswer

\problem Find the center of mass of a two-dimensional plate  %{{{2
that occupies the upper unit semicircle centered at $(0,0)$
and has density
function $y$.
\answer
$\bar x=0$, $\bar y=3\pi/16$
\endanswer

\problem Find the center of mass of a two-dimensional plate  %{{{2
that occupies the upper unit semicircle centered at $(0,0)$
and has density
function $x^2$.
\answer
$\bar x=0$, $\bar y=16/(15\pi)$
\endanswer






\problem Find the area of the surface of a right circular cone of %{{{2
height $h$ and base radius $a$.
\answer
$\pi a\sqrt{h^2+a^2}$
\endanswer

\problem Find the area of the portion of the plane $z=mx$ inside the %{{{2
cylinder $x^2+y^2=a^2$.
\answer
$\pi a^2\sqrt{m^2+1}$
\endanswer

\problem Find the area of the portion of the plane $x+y+z=1$ in the %{{{2
first octant.
\answer
$\sqrt3/2$
\endanswer

\problem Find the area of the upper half of the cone %{{{2
$x^2+y^2=z^2$ inside the cylinder $x^2+y^2-2x = 0$.
\answer
$\pi\sqrt2$
\endanswer

\problem Find the area of the upper half of the cone %{{{2
$x^2+y^2=z^2$ above the interior of one loop of $r=\cos(2\theta)$.
\answer
$\pi\sqrt2/8$
\endanswer

\problem Find the area of the upper hemisphere of  %{{{2
$x^2+y^2+z^2=1$ above the interior of one loop of $r=\cos(2\theta)$.
\answer
$\pi/2-1$
\endanswer

\problem The plane $ax+by+cz=d$ cuts a triangle in the first octant %{{{2
provided that $a, b, c$ and $d$ are all positive.  Set up the
integral to find the area of this triangle.

\problem The surface area formula can be used to compute the surface area %{{{2
of the upper half of the sphere $x^2+y^2+z^2=a^2$, but the integral
is improper.

\begin{itemize}

\item[a.] Set up the appropriate integral to calculate this area and give
    two algebraic reasons why it is improper.

    %% Balof (one will look similar to
    %%  our work from the Implicit Function Theorem).

    \item[b.] Find the surface area of the upper hemisphere of
        $x^2+y^2+z^2=a^2$ above a circle of radius $t$ where $t<a$.

    \item[c.] Find the surface area of the whole upper hemisphere by taking a
        limit of your answer in part (b) as $t$ approaches $a$.

    \end{itemize}





    \problem Evaluate $\DS\int_{0}^{1}\int_{0}^{x}\int_{0}^{x+y}
    2x+y-1 \,dz\,dy\,dx$.
    \answer
    $11/24$
\endanswer

\problem Evaluate $\DS\int_{0}^{2}\int_{-1}^{x^2}\int_{1}^{y} %{{{2
xyz \,dz\,dy\,dx$.
\answer
$623/60$
\endanswer

\problem Evaluate $\DS\int_{0}^{1}\int_{0}^{x}\int_{0}^{\ln y} %{{{2
e^{x+y+z}\,dz\,dy\,dx$.
\answer
$-3e^2/4+2e-3/4$
\endanswer

\problem Evaluate %{{{2
$\DS\int_{0}^{\pi/2}\int_{0}^{\sin\theta}\int_{0}^{r\cos\theta}
r^2\,dz\,dr\,d\theta$.
\answer
$1/20$
\endanswer

\problem Evaluate  %{{{2
$\DS\int_{0}^{\pi}\int_{0}^{\sin\theta}\int_{0}^{r\sin\theta}
r\cos^2\theta\,dz\,dr\,d\theta$.
\answer
$\pi/48$
\endanswer

\problem Evaluate $\DS\int_{0}^{1}\int_{0}^{y^2}\int_{0}^{x+y} %{{{2
x\,dz\,dx\,dy$.
\answer
$11/84$
\endanswer

\problem Evaluate $\DS\int_{1}^{2}\int_{y}^{y^2}\int_{0}^{\ln(y+z)} %{{{2
e^x\,dx\,dz\,dy$.
\answer
$151/60$
\endanswer

\problem For each of the integrals in the previous exercises, give a %{{{2
description of the volume (both algebraic and geometric) that is the
domain of integration.


\problem Find the mass of a cube with edge length 2 and density equal %{{{2
to the square of the distance from one corner.
\answer
$32$
\endanswer

\problem Find the mass of a cube with edge length 2 and density equal %{{{2
to the square of the distance from one edge.
\answer
$64/3$
\endanswer

\problem An object occupies the volume of the upper hemisphere of  %{{{2
$x^2+y^2+z^2=4$ and has density $z$ at $(x,y,z)$. Find the center of mass.
\answer
$\bar x=\bar y=0$, $\bar z=16/15$
\endanswer

\problem An object occupies the volume of the pyramid with corners at  %{{{2
$(1,1,0)$, $(1,-1,0)$, $(-1,-1,0)$, $(-1,1,0)$, and $(0,0,2)$ and has
density $x^2+y^2$ at $(x,y,z)$. Find the center of mass.
\answer
$\bar x=\bar y=0$, $\bar z=1/3$
\endanswer

\problem Verify the moments $M_{xy}$, $M_{xz}$, and $M_{yz}$ %{{{2
of example~\ref{example:3d center of mass} by evaluating the
integrals. 


\problem Find the region $E$ for which $\DS\tint{E} %{{{2
(1-x^2-y^2-z^2) \; \dV $ is a maximum.






\problem Evaluate $\DS\int_{0}^{1}\int_{0}^{x}\int_{0}^{\sqrt{x^2+y^2}} %{{{2
{(x^2+y^2)^{3/2}\over x^2+y^2+z^2}\,dz\,dy\,dx$.
\answer
$\pi/12$
\endanswer

\problem Evaluate $\DS\int\int\int x^2\,\dV $ %{{{2
over the interior of the cylinder $x^2+y^2=1$ between $z=0$ and $z=5$.
\answer
$5\pi/4$
\endanswer

\problem Evaluate $\DS\int\int\int xy\,\dV $ %{{{2
over the interior of the cylinder $x^2+y^2=1$ between $z=0$ and $z=5$.
\answer
$0$
\endanswer

\problem Evaluate $\DS\int\int\int z\,\dV $ %{{{2
over the region above the $x$-$y$ plane, inside $x^2+y^2-2x=0$ and
under $x^2+y^2+z^2=4$.
\answer
$5\pi/4$
\endanswer

\problem Evaluate $\DS\int\int\int yz\,\dV $ %{{{2
over the region in the first octant, inside $x^2+y^2-2x=0$ and 
under $x^2+y^2+z^2=4$.
\answer
$4/5$
\endanswer

\problem Evaluate $\DS\int\int\int x^2+y^2\,\dV $ %{{{2
over the interior of $x^2+y^2+z^2=4$.
\answer
$256\pi/15$
\endanswer

\problem Evaluate $\DS\int\int\int \sqrt{x^2+y^2}\,\dV $ %{{{2
over the interior of $x^2+y^2+z^2=4$.
\answer
$4\pi^2$
\endanswer

\problem Find the mass of a right circular cone of height $h$ and %{{{2
base radius $a$ if the density is proportional to the distance from
the base.
\answer
$\pi kh^2a^2/12$
\endanswer

\problem Find the mass of a right circular cone of height $h$ and %{{{2
base radius $a$ if the density is proportional to the distance from
its axis of symmetry.
\answer
$\pi kha^3/6$
\endanswer

\problem An object occupies the region inside the unit sphere at the %{{{2
origin, and has density equal to the distance from the $x$-axis. Find
the mass.
\answer
$\pi^2/4$
\endanswer

\problem An object occupies the region inside the unit sphere at the %{{{2
origin, and has density equal to the square of the distance from the
origin. Find the mass.
\answer
$4\pi/5$
\endanswer

\problem An object occupies the region between the unit sphere at the %{{{2
origin and a sphere of radius 2 with center at the origin, and has
density equal to the distance from the origin. Find the mass.
\answer
$124\pi/5$
\endanswer





\problem Complete example~\ref{example:change of variables} by %{{{2
converting to polar coordinates and evaluating the integral.

\problem Evaluate $\DS\int\int xy\,dx\,dy$ over the square %{{{2
with corners $(0,0)$, $(1,1)$, $(2,0)$, and $(1,-1)$ in two ways:
directly, and using $x=(u+v)/2$, $y=(u-v)/2$.
\answer
$0$
\endanswer

\problem Evaluate $\DS\int\int x^2+y^2\,dx\,dy$ over the square %{{{2
with corners $(-1,0)$, $(0,1)$, $(1,0)$, and $(0,-1)$ in two ways:
directly, and using $x=(u+v)/2$, $y=(u-v)/2$.
\answer
$2/3$
\endanswer

\problem Evaluate $\DS\int\int (x+y)e^{x-y}\,dx\,dy$ over the triangle %{{{2
with corners $(0,0)$, $(-1,1)$, and $(1,1)$ in two ways:
directly, and using $x=(u+v)/2$, $y=(u-v)/2$.

\problem Evaluate $\DS\int\int y(x-y)\,dx\,dy$ over the parallelogram %{{{2
with corners $(0,0)$, $(3,3)$, $(7,3)$, and $(4,0)$ in two ways:
directly, and using $x=u+v$, $y=u$.

\problem Evaluate $\DS\int\int \sqrt{x^2+y^2}\,dx\,dy$ over the %{{{2
triangle with corners $(0,0)$, $(4,4)$, and $(4,0)$ using $x=u$, $y=uv$.
\answer
$32(\sqrt2+\ln(1+\sqrt2))/3$
\endanswer

\problem Evaluate $\DS\int\int y\sin(xy)\,dx\,dy$ over the %{{{2
region bounded by $xy=1$, $xy=4$, $y=1$, and $y=4$ using
$x=u/v$, $y=v$.
\answer
$3\cos(1)-3\cos(4)$
\endanswer


\def\dint#1{\int\int_{#1}}
\textbf{Vector Calculus}



Sketch the vector fields; check your work with Maple's \verb|fieldplot|
command. 

\problem $\langle x,y\rangle$  %{{{2

\problem $\langle -x, -y\rangle$  %{{{2

\problem $\langle x,-y\rangle$  %{{{2

\problem $\langle \sin x,\cos y\rangle$  %{{{2

\problem $\langle y,1/x\rangle$  %{{{2

\problem $\langle x+1,x+3\rangle$  %{{{2

\problem Verify equation~\ref{eq:inverse square field as gradient}. %{{{2





\problem Compute $\DS\int_{\partial D} 2y\,dx + 3x\,dy$,  %{{{2
where $D$ is described by $0\le x,y\le 1$.
\answer
$1$
\endanswer

\problem Compute $\DS\int_{\partial D} xy\,dx + xy\,dy$,  %{{{2
where $D$ is described by $0\le x,y\le 1$.
\answer
$0$
\endanswer

\problem Compute $\DS\int_{\partial D} e^{2x+3y}\,dx + e^{xy}\,dy$,  %{{{2
where $D$ is described by $-2\le x\le 2$, $-1\le y\le 1$.
\answer
$1/(2e)-1/(2e^7)+e/2-e^7/2$
\endanswer

\problem Compute $\DS\int_{\partial D} y\cos x\,dx + y\sin x\,dy$,  %{{{2
where $D$ is described by $0\le x\le \pi/2$, $1\le y\le 2$.
\answer
$1/2$
\endanswer

\problem Compute $\DS\int_{\partial D} x^2y\,dx + xy^2\,dy$,  %{{{2
where $D$ is described by $0\le x\le 1$, $0\le y\le x$.
\answer
$-1/6$
\endanswer

\problem Compute $\DS\int_{\partial D} x\sqrt{y}\,dx + \sqrt{x+y}\,dy$,  %{{{2
where $D$ is described by $1\le x\le 2$, $2x\le y\le 4$.
\answer
$(2\sqrt3-10\sqrt5+8\sqrt6)/3-2\sqrt2/5+1/5$
\endanswer

\problem Compute $\DS\int_{\partial D} (x/y)\,dx + (2+3x)\,dy$,  %{{{2
where $D$ is described by $1\le x\le 2$, $1\le y\le x^2$.
\answer
$11/2-\ln(2)$
\endanswer

\problem Compute $\DS\int_{\partial D} \sin y\,dx + \sin x\,dy$,  %{{{2
where $D$ is described by $0\le x\le \pi/2$, $x\le y\le \pi/2$.
\answer
$2-\pi/2$
\endanswer

\problem Compute $\DS\int_{\partial D} x\ln y\,dx$, %{{{2
where $D$ is described by $1\le x\le 2$, $\DS e^x\le y\le e^{x^2}$.
\answer
$-17/12$
\endanswer

\problem Compute $\DS\int_{\partial D} \sqrt{1+x^2}\,dy$,  %{{{2
where $D$ is described by $-1\le x\le 1$, $x^2\le y\le 1$.
\answer
$0$
\endanswer

\problem Compute $\DS\int_{\partial D} x^2y\,dx - xy^2\,dy$,  %{{{2
where $D$ is described by $x^2+y^2\le 1$.
\answer
$-\pi/2$
\endanswer

\problem Compute $\DS\int_{\partial D} y^3\,dx + 2x^3\,dy$,  %{{{2
where $D$ is described by $x^2+y^2\le 4$.
\answer
$12\pi$
\endanswer

\problem Finish our proof of Green's Theorem by showing that %{{{2
$\DS\oint_C Q\,dy=\dint{D} {\partial Q\over\partial x}\,dA$.






\problem Let $\vF =\langle xy,-xy\rangle$ and  %{{{2
let $D$ be given by $0\le x\le 1$, $0\le y\le 1$.
Compute $\DS\int_{\partial D} \vF \cdot d\vr $ and
$\DS\int_{\partial D} \vF \cdot\vN \,ds$.
\answer
$-1$, $0$
\endanswer

\problem Let $\vF =\langle ax^2,by^2\rangle$ and  %{{{2
let $D$ be given by $0\le x\le 1$, $0\le y\le 1$.
Compute $\DS\int_{\partial D} \vF \cdot d\vr $ and
$\DS\int_{\partial D} \vF \cdot\vN \,ds$.
\answer
$0$, $a+b$
\endanswer

\problem Let $\vF =\langle ay^2,bx^2\rangle$ and  %{{{2
let $D$ be given by $0\le x\le 1$, $0\le y\le x$.
Compute $\DS\int_{\partial D} \vF \cdot d\vr $ and
$\DS\int_{\partial D} \vF \cdot\vN \,ds$.
\answer
$(2b-a)/3$, $0$
\endanswer

\problem Let $\vF =\langle \sin x\cos y,\cos x\sin y\rangle$ and  %{{{2
let $D$ be given by $0\le x\le \pi/2$, $0\le y\le x$.
Compute $\DS\int_{\partial D} \vF \cdot d\vr $ and
$\DS\int_{\partial D} \vF \cdot\vN \,ds$.
\answer
$0$, $1$
\endanswer

\problem Let $\vF =\langle y,-x\rangle$ and  %{{{2
let $D$ be given by $x^2+y^2\le 1$.
Compute $\DS\int_{\partial D} \vF \cdot d\vr $ and
$\DS\int_{\partial D} \vF \cdot\vN \,ds$.
\answer
$-2\pi$, $0$
\endanswer

\problem Let $\vF =\langle x,y\rangle$ and  %{{{2
let $D$ be given by $x^2+y^2\le 1$.
Compute $\DS\int_{\partial D} \vF \cdot d\vr $ and
$\DS\int_{\partial D} \vF \cdot\vN \,ds$.
\answer
$0$, $2\pi$
\endanswer






\problem Find the area of the portion of $x+2y+4z=10$ in the first %{{{2
octant.
\answer
$25\sqrt{21}/4$
\endanswer

\problem Find the area of the portion of $2x+4y+z=0$ %{{{2
inside $x^2+y^2=1$.
\answer
$\pi\sqrt{21}$
\endanswer

\problem Find the area of $z=x^2+y^2$ that lies below $z=1$. %{{{2
\answer
$\pi(5\sqrt5-1)/6$
\endanswer

\problem Find the area of $z=\sqrt{x^2+y^2}$ that lies below $z=2$. %{{{2
\answer
$4\pi\sqrt2$
\endanswer

\problem Find the area of the portion of $x^2+y^2+z^2=a^2$ that lies %{{{2
in the first octant.
\answer
$\pi a^2/2$
\endanswer

\problem Find the area of the portion of $x^2+y^2+z^2=a^2$ that lies %{{{2
above $x^2+y^2\le b^2$.
\answer
$2\pi a(a-\sqrt{a^2-b^2})$
\endanswer

\problem Find the area of $z=x^2-y^2$ that lies inside $x^2+y^2=a^2$. %{{{2
\answer
$\pi((1+4a^2)^{3/2}-1)/6$
\endanswer

\problem Find the area of $z=xy$ that lies inside $x^2+y^2=a^2$. %{{{2
\answer
$2\pi((1+a^2)^{3/2}-1)/3$
\endanswer

\problem Find the area of $x^2+y^2+z^2=a^2$  %{{{2
that lies above the interior of the circle given in polar coordinates
by $r=a\cos \theta$.
\answer
$\pi a^2-2a^2$
\endanswer

\problem Find the area of the cone $z=k\sqrt{x^2+y^2}$ %{{{2
that lies above the interior of the circle given in polar coordinates
by $r=a\cos \theta$.
\answer
$\pi a^2\sqrt{1+k^2}/4$
\endanswer

\problem Find the area of the plane $z=ax+by+c$ that lies over a %{{{2
region $D$ with area $A$.
\answer
$A\sqrt{1+a^2+b^2}$
\endanswer

\problem Find the area of the cone $z=k\sqrt{x^2+y^2}$ that lies over a %{{{2
region $D$ with area $A$.
\answer
$A\sqrt{k^2+1}$
\endanswer

\problem Find the area of the cylinder $x^2+z^2=a^2$ that lies inside %{{{2
the cylinder $x^2+y^2=a^2$.
\answer
$8a^2$
\endanswer

\problem The surface $f(x,y)$ can be represented with the vector %{{{2
function $\langle x,y,f(x,y)\rangle$. Set up the surface area integral using
this vector function and compare to the integral of
section~\ref{sec:surface area}. 





\problem Find the center of mass of an object that occupies the upper %{{{2
hemisphere of $x^2+y^2+z^2=1$ and has density $x^2+y^2$.
\answer
$(0,0,3/8)$
\endanswer

\problem Find the center of mass of an object that occupies the %{{{2
surface $z=xy$, $0\le x,y\le 1$, and has density $\sqrt{1+x^2+y^2}$.
\answer
$(11/20,11/20,3/10)$
\endanswer

\problem Find the centroid of the surface of a right circular cone of %{{{2
height $h$ and base radius $r$, not including the base.
\answer
on center axis, $h/3$ above the base
\endanswer

\problem Evaluate $\DS \dint{D} \langle 2,-3,4\rangle\cdot {\bf %{{{2
N}\,dS$, where $D$ is given by $z=x^2+y^2$, $-1\le x\le 1$, $-1\le
y\le 1$, oriented up.
\answer
$16$
\endanswer

\problem Evaluate $\DS \dint{D} \langle x,y,3\rangle\cdot {\bf %{{{2
N}\,dS$, where $D$ is given by $z=3x-5y$, $1\le x\le 2$, $0\le
y\le 2$, oriented up.
\answer
$7$
\endanswer

\problem Evaluate $\DS \dint{D} \langle x,y,-2\rangle\cdot {\bf %{{{2
N}\,dS$, where $D$ is given by $z=1-x^2-y^2$, $x^2+y^2\le1$,
oriented up.
\answer
$-\pi$
\endanswer

\problem Evaluate $\DS \dint{D} \langle xy, yz,zx\rangle\cdot {\bf %{{{2
N}\,dS$, where $D$ is given by $z=x+y^2+2$, $0\le x\le 1$, $x\le
y\le 1$, oriented up.
\answer
$-137/120$
\endanswer

\problem Evaluate $\DS \dint{D} \langle e^x, e^y,z\rangle\cdot {\bf %{{{2
N}\,dS$, where $D$ is given by $z=xy$, $0\le x\le 1$, $-x\le
y\le x$, oriented up.
\answer
$-2/e$
\endanswer

\problem Evaluate $\DS \dint{D} \langle xz,yz,z\rangle\cdot {\bf %{{{2
N}\,dS$, where $D$ is given by $z=a^2-x^2-y^2$, $x^2+y^2\le b^2$, 
oriented up.
\answer
$\pi b^2(-4b^4-3b^2+6a^2b^2+6a^2)/6$
\endanswer





\problem %{{{2
Let $\vF =\langle z,x,y\rangle$.
The plane $z=2x+2y-1$ and the paraboloid $z=x^2+y^2$ intersect in a
closed curve. Stokes's Theorem implies that
$$\dint{D_1} (\nab\times\vF )\cdot \vN \,dS=
\oint_C \vF \cdot d\vr =
\dint{D_2} (\nab\times\vF )\cdot \vN \,dS,
$$
where the line integral is computed over the intersection $C$ of the plane
and the paraboloid, and the two surface integrals are computed over
the portions of the two surfaces that have boundary $C$ (provided, of
course, that the orientations all match). Compute all three integrals.
\answer
$-3\pi$
\endanswer

\problem Let $D$ be the portion of $z=1-x^2-y^2$ above the $x$-$y$ %{{{2
plane, oriented up, and let $\vF =\langle
xy^2,-x^2y,xyz\rangle$. Compute $\DS\dint{D} (\nab\times{\bf
F})\cdot \vN \,dS$.
\answer
$0$
\endanswer

\problem Let $D$ be the portion of $z=2x+5y$ inside $x^2+y^2=1$, %{{{2
oriented up, and
let $\vF =\langle y,z,-x\rangle$. Compute
$\DS\int_{\partial D} \vF \cdot d\vr $.
\answer
$-4\pi$
\endanswer

\problem Let $D$ be the portion of $z=px+qy+r$ over a region in the %{{{2
$x$-$y$ plane that has area $A$, oriented up, and 
let $\vF =\langle ax+by+cz,ax+by+cz,ax+by+cz\rangle$. Compute
$\DS\int_{\partial D} \vF \cdot d\vr $.
\answer
$A(p(c-b)+q(a-c)+a-b)$
\endanswer

\problem Let $D$ be any surface and  %{{{2
let $\vF =\langle P(x),Q(y),R(z)\rangle$ ($P$ depends only on $x$,
$Q$ only on $y$, and $R$ only on $z$). Show that
$\DS\int_{\partial D} \vF \cdot d\vr =0$.

\problem Show that $\DS\int_C f\nab g+g\nab f\cdot d\vr =0$, where %{{{2
$\bf r$ describes a closed curve $C$ to which Stokes's Theorem
applies.


\end{multicols}

%%% Local Variables: 
%%% mode: latex
%%% TeX-master: "free234"
%%% End: 
