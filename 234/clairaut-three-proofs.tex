\documentclass{amsbook}

\input ../preamble
\input ../free-macros
\input ../gans.tex

\begin{document}
\begin{center}
  \Large\color{badgerred}
  Three proofs of the converse to Clairaut's theorem
\end{center}
\bigskip 

If $P(x,y)$ and $Q(x,y)$ are two functions on a convex domain that satisfy 
\[
\pdd Qx = \pdd Py \tag{$\star$}
\]
then there is a function $f(x,y)	$ whose partial derivatives are
\[
 \pdd fx = P(x,y) \text{ and } \pdd fy = Q(x,y). 
\]

\section{First proof -- differentiating under the integral}
We claim that the function given by
\[
f(x,y) = \int_0^x P(u, 0) du + \int_0^y Q(x, v) dv
\tag{\dag}
\]
does the trick.

To verify this we compute the partial derivatives of this function.  The $y$ derivative follows directly from the fundamental theorem of calculus:
\begin{align*}
f_y(x, y) &= \pdd{}y \left\{ \int_0^x P(u, 0) du + \int_0^y Q(x, v) dv \right\}\\
&= 0+\pdd{}y \int_0^y Q(x, v) dv\\
&= Q(x, y).
\end{align*}
For the $x$-derivative we argue as follows:
\begin{align*}
f_x(x, y) &= P(x,0) + \pdd{}x  \int_0^y Q(x, v) dv &\carefulnow \\
&= P(x,0) + \int_0^y Q_x(x, v) dv &\carefulnow\\
&= P(x,0) + \int_0^y P_y(x, v) dv \\
&= P(x,0) + \bigl[P(x, y) - P(x, 0)\bigr] \\
&= P(x, y).
\end{align*}
To go from the first to the second line in this computation we used
\[
  \pdd{}x  \int_0^y Q(x, v) dv = \int_0^y \pdd Qx(x, v) dv,
\]
which is known as ``differentiating under the integral.''  This is justified, but the proof that differentiating under the integral is allowed is not simple.

\section{Second proof -- switching a double integral}
Our second proof starts again with the claim that 
\[
 f(x,y) = \int_0^x P(u, 0) du + \int_0^y Q(x, v) dv 
\]
is the function we are looking for.  The $y$ derivative again follows as in the first proof. To compute the $x$ derivative we use a double integral to rewrite our definition of $f(x,y)$ as follows
\begin{align*}
f(x, y) &=\int_0^x P(u, 0) du + \int_0^y Q(x, v) dv \\
&= \int_0^x P(u,0)du + \int_0^y \left\{Q(0, v) dv + \int_0^x \pdd Qx(u,v) du\right\} dv \\
&= \int_0^x P(u,0)du + \int_0^y Q(0, v) dv + \int_0^y\int_0^x \pdd Py(u,v) du\, dv  &\text{\dfnt(switch order}\\
&= \int_0^x P(u,0)du + \int_0^y Q(0, v) dv + \int_0^x\int_0^y \pdd Py(u,v) dv\, du  &\text{\dfnt of integration)}\\
&= \int_0^x P(u,0)du + \int_0^y Q(0, v) dv + \int_0^x\int_0^y \bigl[ P(u,v)\bigr]_{v=0}^y du\\
&= \int_0^x P(u,0)du + \int_0^y Q(0, v) dv + \int_0^x\int_0^y  \bigl[P(u,y)-P(u,0)\bigr] du \\
&= \int_0^x P(u,y)du + \int_0^y Q(0, v) dv . 
\end{align*}
This new form is easy to differentiate with respect to $x$:
\[
 \pdd fx = 
 \pdd{}x\left\{ \int_0^x P(u,y)du + \int_0^y Q(0, v) dv
 \right\} 
 =P(x,y).
\]

\section{The third proof -- using Green's theorem}
Our last proof also rewrites the definition of $f(x,y)$ to make it easier to differentiate with respect to $x$.  By applying Green's theorem to the rectangle 
\[
\cR = \{ (u,v) : 0\leq u\leq x, 0\leq v\leq y\}
\]
we get 
\[
\oint_\cC P(x, y)dx+Q(x, y)dy
=\liint_\cR \left\{\pdd Qx-\pdd Py\right\} dA
=0
\]
where $\cC$ is the counterclockwise traversed boundary of the rectangle $\cR$.  Parametrizing the edges of $\cR$ we find that the line integral is
\begin{multline*}
\oint_\cC P(x, y)dx+Q(x, y)dy\\
=\int_0^x P(u,0)du + \int_0^y Q(x, v)dv - \int_0^x P(u, y) du - \int_0^y Q(0, v)dv.
\end{multline*}
Therefore we have
\[
  f(x,y) 
  = \int_0^x P(u,0)du + \int_0^y Q(x, v)dv = \int_0^x P(u, y) du + \int_0^y Q(0, v)dv.
\]
To find $f_x$ we differentiate the second form, and to find $f_y$ we differentiate the first.
\end{document}




