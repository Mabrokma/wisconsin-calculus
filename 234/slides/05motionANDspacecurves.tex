%% Time-stamp:  Wed Jun 12 16:35:21 2013
\documentclass{amsart}
\input ../../preamble.tex
\begin{document}


\section{Vectors}

\subsection{Geometric description of vectors}
\label{sec:geometric-description-of-vectors}
A vector is an arrow connecting two points.  If the points are $A$ and $B$ then we
call the vector $\tpv AB$. If we translate a vector $\tpv AB$ \emph{without turning
it} then we say that the resulting vector $\tpv CD$ is the same vector as the
original vector $\tpv AB$.
\begin{figure}[h]\centering
  \input ../figures/005equalvectors.tex
  \caption{$\tpv AB$ and $\tpv CD$ represent the same vector}
\end{figure}

\subsection{Arithmetic of vectors}
\label{sec:arithmetic-of-vectors}
To add two vectors $\tpv AB$ and $\tpv PQ$ we first translate the vector $\tpv PQ$ so
that its initial point becomes $B$; let the result of this translation be the vector
$\tpv BC$.  Then the sum of $\tpv AB$ and $\tpv PQ$ is $\tpv AC$: in a formula,
\[
\tpv AB + \tpv PQ = \tpv AB + \tpv BC = \tpv AC.
\]

\begin{figure}[h]
  \def\figfont{\sffamily\footnotesize\color{darkbluegreen}\centering}
  \def\addingvectorsCapA{\parbox{1in}{\figfont%
  to add\\
  two vectors\dots}}
  \def\addingvectorsCapB{\parbox{1in}{\figfont%
  \dots move one vector until its initial point\dots}}
  \def\addingvectorsCapC{\parbox{1in}{\figfont%
  \dots is the end point of the other\dots} }
  \def\addingvectorsCapD{\parbox{1in}{\figfont%
  \dots and combine them.} }
\input figures/005adding-vectors.pdf_tex
\end{figure}
A different way of adding two vectors $\tpv AB$ and $\tpv PQ$ is to move the vectors
around until they have the same initial point.  Two vectors with a common initial
point form two sides of a parallelogram (see
Figure~\ref{fig:adding-vectors-parallelogram}) and the sum of the two vectors is the
diagonal of that parallelogram.
\begin{figure}[h]
  \input figures/005adding-vectors-parallelogram.pdf_tex
  \caption{{\bfseries Using a parallelogram to add vectors. }
  To find $\tpv AB  + \tpv AD$ we move the vector $\tpv AD$ so that its initial point
  is at $B$, i.e.~the endpoint of $\tpv AB$.  This gives us a 
  parallelogram $ABCD$, where $\tpv AD = \tpv BC$.  Therefore $\tpv AB+\tpv AD = \tpv
  AB+\tpv BC = \tpv AC$ }
  \label{fig:adding-vectors-parallelogram}
\end{figure}





\end{document}

