% !TEX Program = XeLaTeX
% Time-stamp: Sat Jan  4 04:56:21 2014
\documentclass[11pt]{amsbook}

\input ../preamble
\input ../free-macros
\input ../gans.tex

\newcommand\version{0.9}
\newcommand\semester{Spring 2014}

\begin{document}
\begin{center}
  \bfseries\color{badgerred}
  Comments on the MIU math 234 text
  \\
  \mdseries\itshape
  January 2014
\end{center}
The MIU calculus texts were produced as part of the math department's MIU project,
(which allowed us to hire three faculty).

The overall goal has been to produce a text that can be used to allow students to
learn third semester calculus by understanding what is going on rather than by trying
to memorize a list of mysterious formulas.  If you are used to teaching out of
Thomas\&co, then you will find that the math 234 text is different in some aspects.
The reasons for those differences are explained below.   The choices were made after
consultation with faculty in a large number of client departments (stats, econ,
zoology, chemistry, college of engineering).  
The text is a work in
progress and any contributions, suggestions, corrections you may have are welcome.

\section*{vectors}
Vectors is an orphaned topic.  It used to be part of the calculus AP/BC exam,
but five or six years ago the college board cut it from the AP calc curriculum. 
We still have to give math 222 credit to high school students who pass the AP/BC
exam, and for consistency we also give 222 transfer credit to students who took
calc 2 somewhere, even if they didn't cover vectors.  So there will be a group
of students in math 234 who do not know what a dot product is.  That's why the
notes have a chapter 1 on vectors.  It will be review for most students, but
others will really need it.  

The second chapter on parametric curves is there partially to practice vectors
(problem 4, page 32 on finding the tangent to the twisted cubic, and seeing
where it intersects the $xy$ plane will separate the beginners from the advanced
students.) 

\section*{functions}
In one variable calculus students have already seen examples of functions (ad
nauseum).  For more than one variable they haven't, so we do some here, mainly
linear and quadratic functions.

Quadratic functions will show up again later in the 2nd derivative test.  The
traditional way of presenting the 2nd derivative test boils down to memorizing
something about the signs of $A$ and $AC-B^2$, where $A$, $B$, $C$ are the 2nd
derivatives at a critical point.  Students do not understand why it works, and in the
long run will not retain anything of value from this.  Instead we explain the 2nd
order Taylor expansion, and classify the possible 2nd order terms.  Since the
students don't know linear algebra we stick to quadratic forms in two variables which
you can classify by completing the square.  This square completion is done in chapter
3.

\section*{partial derivatives}
The text spends almost no time doing anything like $\epsilon$-$\delta$ proofs or
definitions on continuity.  Some of this is necessary, but I don't think math 234 is
the place to study functions that are not Frechet differentiable even though they
have partial derivatives everywhere.  

The key concept is the linear approximation formula
\[
  f(x+\Delta x, y+\Delta y) = f(x,y) + \pdd fx(x,y)\,\Delta x + \pdd fy (x,y)\,
  \Delta y + \dots
\]
which says that when $x$ and $y$ change, the change in
$f(x, y)$ consists of two terms, one caused by the change in $x$, and the other by
the change in $y$.  
\begin{itemize}
\item If students understand this then the same idea should help them understand
  the chain rule for differentiating $f(x(t), y(t))$;
\item students should know how to take the second derivative of $f(x(t), y(t))$,
  without relying on unexplained ``derivative trees.''
\item If you write the formula in vector form you can identify the gradient as
  direction of steepest increase;
\end{itemize}
The standard textbooks introduce ``directional derivatives'' and use the
notation $D_\vu f(x)$, where $\vu$ is a unit vector.  It stands for
$\vu\dpp\nabla f(x) = \frac{d}{dt}f(\vx+t\vu)|_{t=0}$.  It is like the
manifolds definition of the differential $df(x)$ as an element of the cotangent
space, but you don't really need it anywhere, and it just ends up being another
formula students memorize without a lot of understanding.

\section*{maxima and minima}
You can't really prove that every function attains its max\&min on a compact set,
but by examples you can sort of explain the statement.  If a function is given
as the product of a bunch of simple functions then its zero set is easy to
describe, and students should able to explain in words why self intersections of
the zero set are critical points, and why each bounded component of the region
$\{f\neq0\}$ contains at least one critical point.  In the second derivative test I
asked the students to classify the second order terms into definite or
indefinite, and for saddle points factor the quadratic part to identify the two
tangent lines to the level set (all this to show that they understood something
of what was going on and that they had not memorized some random procedure).

\section*{Lagrange multipliers}
The econ department is a big fan.  The text contains a
bold lie, namely that there is no interpretation of the Lagrange multiplier.  I
will fix that in future versions of the text.

\section*{integrals}
Setting up an integral is at least as important as computing one, so I think the
section on ``densities'' and the examples about moments of inertia are important.
But they shouldn't be formulas that students memorize.  E.g.~students should not
memorize that the moment of inertia about the $z$-axis is $I_z=\frac12\iiint
(x^2+y^2)dV$, and then be asked to compute that integral;  instead they should know
how to derive the formula for $I_z$ from basic principles.

Engineers have told us that students coming out of 234 can do fancy triple integrals,
but they don't know how to combine all the kinetic energies of small pieces of a
moving solid into an integral.  I.e., they can compute a triple integral but they
don't understand what the answer means.

\section*{vector calculus}
Again, by going through the material at great
speed one can force the students to memorize a lot of formulas for the various
integrals.  At the end they should however also know how to compute the work or flux
integrals of a constant vector field across a line segment, without having to set up
a parametrization for the line segment (I think that very simple question tests if
the students actually understood anything).

This chapter is not in final form, and could use more worked problems, which I will add
to a future version.  For now there are some sample problems posted at  
\begin{center}
  \footnotesize
  \url{http://www.math.wisc.edu/~angenent/234.2013f/final/reviewproblems.html}
\end{center}

Our client departments expect that students will get a ``superficial understanding,''
or ``a first introduction to'' vector calculus in math 234.  Students in the physical
sciences and engineering who need more should take math 321 (I say this in lecture).
\end{document}
% vim: fdc=2 fdm=expr 
