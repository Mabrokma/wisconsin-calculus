



\project{Models for the dynamics of populations}

\section{Purpose of this project}

We will build mathematical models for the rates of change of different
populations from first principle arguments.  We will see that the same
mathematical model can be used to describe such disparate things as
populations of living organisms and chemical concentrations.

\section{Malthusian Population Growth}

Consider a population living in isolation with an abundance of
resources.  Let $P(t)$ denote the size of the population at some time,
$t$.  Note that $P(0)$ is the size of the population at time zero
(where we can define ``time zero'' to be whenever we want: 0 A.D.,
January 1st 2012, yesterday, etc.).  We now make the
\textit{assumption} that for any small $h>0$, representing a small
window of time, the growth of the population from time $t$ to time
$t+h$ should be proportional to both
\begin{enumerate}
\item the size of the population, and
\item the size of $h$.
\end{enumerate}
Mathematically, we assume that
\begin{equation}\label{main:eq}
  P(t+h) \approx P(t) + \lambda P(t) h,
\end{equation}
for some $\lambda >0$ (the constant of proportionality).  Rearranging
terms yields
\begin{equation}\label{eq:main2}
  \frac{P(t+h)-P(t)}{h} \approx \lambda P(t).
\end{equation}
The term $\lambda$ can often be approximated from experimental data
(How?  The project for Chapter 6 will explain this in detail.).  Note
that the right hand side of equation \eqref{eq:main2} does not depend
upon $h$.  Further, the above is assumed to hold for \textit{all}
$h>0$, and so letting $h$ get very small we see that the left hand
side is well approximated by the derivative of $P$ at $t$ (see page 20
of the notes):
\[
P'(t) = \lim_{h \to 0}\frac{P(t+h)-P(t)}{h}.
\]
Thus, we are led to the conclusion that the rate of change of $P(t)$
equals $\lambda P(t)$ for some $\lambda > 0$.  Mathematically, we have
that $P$ satisfies the {\em differential equation}
\begin{equation*}
  P'(t) = \lambda P(t).
\end{equation*}
Note that this conclusion was reached logically from our assumptions
above, which we consider to be our ``first principles'' for this
model.  If you do not believe one of the assumptions holds for a
particular population, then you can not trust the conclusion!

We will not consider how to solve differential equations here (you
will see these again in Math 222), however we observe that the above
equation makes a startling claim:


\begin{center}
  {\em the population will increase without bound forever!}
\end{center}

Let's see why by returning to equation \eqref{main:eq}.  For ease, we
suppose that $P(0) = 1$, $\lambda = 1$, and we take $h = 1/2$.  Using
these values, fill in the following chart:

\begin{center}
  \begin{tabular}{|c|c|}
    \hline
    $t$ & $P_{1/2}(t)$\\ \hline
    0 & 1.00 \\ \hline
    0.5 & 1.50  \\ \hline
    1.0 & 2.25 \\ \hline
    1.5 & \\ \hline
    2.0 & \\ \hline
    2.5 & \\ \hline
    3.0 & \\ \hline
  \end{tabular}
\end{center}
where we are writing $P_{1/2}(t)$ for the size of the population since
we are explicitly choosing an $h$ of size $1/2$.  Note that the
population is not just growing, but the {\em rate} at which the
population is growing is even increasing.

\section{Problems}

\problemfont
\problem
Make a similar chart as that above with $h = 1/4$.  That is,
calculate $P_{1/4}(t)$ for $t \in \{0,0.25, 0.5, 0.75, \dots,
3.0\}$. How do the solutions $P_{1/4}(t)$ and $P_{1/2}(t)$ compare
for
\[
t \in \{0.5, 1, 1.5, 2, 2.5, 3.0\}?
\]
Why do you think this is happening?
\noproblemfont

\section{The Logistic Growth Model}

The fact that the previous model predicts continued growth, without
bound, is a serious shortcoming for many reasonable situations.  For
example, if there is a limited amount of natural resources,
unsustained growth is clearly not realistic.  The following twist on
the previous model fixes the problem.

We still consider a small time window $[t,t+h]$ and ask how the
population changes over that time period.  Now we make the following
assumptions:
\begin{enumerate}
\item There is a ``carrying capacity'' $K>0$ such that if $X(t) > K$,
  the population should decrease and if $X(t) < K$, then the
  population should increase.
\item The further from $K$ the population is, the stronger the
  gain/decrease.
\end{enumerate}

One such model satisfying these reasonable ideas is
\begin{equation}\label{eq:Logistic}
  X(t + h) \approx X(t) + \lambda X(t)\left(1 - \frac{X(t)}{K}\right) h.
\end{equation}
Similarly to the Malthusian model, the associated {\em differential
equation} satisfied by $X$ is
\begin{equation}\label{eq:logistic_ode}
  X'(t) = \lambda X(t) \left(1 - \frac{X(t)}{K}\right).
\end{equation}
The model described here is called the {\em logistical growth
model}.

Returning to equation \eqref{eq:Logistic}, complete the following
charts for the different $X(0)$ assuming that $\lambda = 1$, $K = 4$,
and $h = 0.5$.

\begin{center}
  \begin{tabular}{|c|c|}
    \hline
    $t$ & $X(t)$\\ \hline
    0 & 1.00   \\ \hline
    0.5 & 1.375   \\ \hline
    1.0 & 1.826  \\ \hline
    1.5 &\\ \hline
    2.0 & \\ \hline
    2.5 & \\ \hline
    3.0 & \\ \hline
  \end{tabular}\qquad 
  \begin{tabular}{|c|c|}
    \hline
    $t$ & $X(t)$\\ \hline
    0 & 7.00  \\ \hline
    0.5 & 4.375  \\ \hline
    1.0 & 4.170 \\ \hline
    1.5 &\\ \hline
    2.0 & \\ \hline
    2.5 & \\ \hline
    3.0 & \\ \hline
  \end{tabular}
\end{center}

What do you think each $X(t)$ converge to for larger and larger $t$?



\bigskip

\problemfont
\problem What do you think each $X(t)$ converge to for larger and
larger $t$?  Why?

\problem More generally, consider $X(t)$ that satisfies
\eqref{eq:logistic_ode} for some $\lambda> 0$ and $K>0$.  What is the
sign of $X'(t)$ when $X(t) > K$?  What about when $X(t) < K$?  Using
these facts, argue what will happen to $X(t)$ as $t$ gets very large.
Try to be persuasive.  (Note: proving this predicted behavior is a
subject for a future mathematic course on \textit{dynamical systems}.)

\problem Check that that if $X$ satisfies \eqref{eq:Logistic}, then it
satisfies the assumptions $(a)$ and $(b)$ above.

\problem If $X$ satisfies \eqref{eq:Logistic} for all $h>0$, why
should it satisfy the differential equation \eqref{eq:logistic_ode}?

\noproblemfont
\bigskip


\section{Chemistry}

We now change topics and consider the chemical reaction
\begin{equation*}
  A + B \to 2A.
\end{equation*}
Compare this with the chemical reaction found on page 21 of the notes.
Note that every time the above chemical reaction occurs, we gain a
molecule of $A$ and lose a molecule of $B$.  This implies that the
total number of molecules, $A + B$, is {\em conserved}.  Thus, if
$[A](t)$ and $[B](t)$ represent the concentration of $A$ and $B$ at
time $t$, then
\[
[A](t) + [B](t) = M
\]
implying
\begin{equation}\label{eq:noB}
  [B](t) = M-[A](t).
\end{equation}
We now make the following assumptions, which are our {\em first
principles}. The number of times the reaction happens in the the
time window $[t,t+h]$ is proportional to
\begin{enumerate}
\item the product of $[A]$ and $[B]$, and
\item the size of $h$.
\end{enumerate}

\section{Problems}

\problemfont
\problem\label{ex:first_princ_chem} Use the first principles above to
find a difference equation, similar to \eqref{eq:Logistic}, for
$[A](t)$.  Note that, due to \eqref{eq:noB}, $[B](t)$ should not
appear in the equation.
%Thus, letting $\kappa>0$ denote the constant of proportionality,
\begin{align*}
  [A](t+h) = [A](t)
  +% \kappa [A](t)[B](t) = [A](t) + \kappa [A](t)(M - [A](t)).
\end{align*}


\problem Using your result from Exercise \ref{ex:first_princ_chem},
derive a differential equation for $[A](t)$:
% \[
%	[A](t+h) = [A](t) + \frac{\kappa}{M} [A](t)\left(1 -
%   \frac{[A](t)}{M}\right).
% \]
% Solving similarly as before,
\[
\frac{d[A](t)}{dt}
=% \frac{\kappa}{M} [A](t)\left(1 - \frac{[A](t)}{M}\right),
\]
Compare the differential equation to \eqref{eq:logistic_ode}.
\noproblemfont

\bigskip

\section*{Report Instructions} Using complete sentences, write a few
paragraphs summarizing the project.  Next, write up your solutions to
all of the exercises.  Be sure to include all pertinent information
from the project itself.  That is, a reader should be able to sit down
with only your report and be able to understand what you are doing,
and the conclusions you have drawn.


