% Time-stamp: Aug 22 11:39 syllabus.tex
\documentclass{amsproc}
\begin{document}
\title{MIU-Math 221 Lecture schedule}
\maketitle
\section*{Week 1}
\begin{itemize}
\item \textbf{Chapter 1. Numbers and Functions}
\item What is a number?
\item Functions
\item Implicit functions
\item Inverse functions
\item Inverse trigonometric functions
\end{itemize}

\section*{Week 2}
\begin{itemize}
\item \textbf{Chapter 2. Derivatives}
\item The tangent to a curve
\item An example -- tangent to a parabola
\item Instantaneous velocity
\item Rates of change
\item Examples of rates of change
\end{itemize}

\section*{Week 3}
\begin{itemize}
\item \textbf{Chapter 3.  Limits and continuous functions}
\item Informal definition of limits
\item ``The formal, authoritative, definition of limit''
\item Variations on the limit theme
\item Properties of the Limit
\item Examples of limit computations
\item When limits fail to exist
\item Limits that equal $\infty$
\item What's in a name? -- Free Variables and Dummy variables
\item Limits and Inequalities
\item Continuity
\item Substitution in Limits
\end{itemize}

\section*{Week 4}
\begin{itemize}
\item Two Limits in Trigonometry
\item Asymptotes
\item \textbf{Chapter 4.  Derivatives.}
\item Derivatives Defined
\item Direct computation of derivatives
\item Differentiable implies Continuous
\item Some non-differentiable functions
\end{itemize}

\section*{Week 5}
\begin{itemize}
\item The Differentiation Rules
\item Differentiating powers of functions
\item Higher Derivatives
\item Differentiating Trigonometric functions
\item The Chain Rule
\item Implicit differentiation
\end{itemize}

\section*{Week 6}
\begin{itemize}
\item \textbf{Chapter 5. Graphsketching an Max-min problems}
\item Tangent and Normal lines to a graph
\item The Intermediate Value Theorem
\item Finding sign changes of a function
\item Increasing and decreasing functions
\item Examples
\item Maxima and Minima
\item Must a function always have a maximum?
\item Examples -- functions with and without maxima or minima
\end{itemize}

\section*{Week 7}
\begin{itemize}
\item General method for sketching the graph of a function
\item Convexity, Concavity and the Second Derivative
\item Optimization Problems
\item Parametrized Curves
\end{itemize}

\section*{Week 8}
\begin{itemize}
\item l'Hopital's rule
\item \textbf{Chapter 7. Exponentials and Logarithms}
\item Exponents
\item Logarithms
\item Properties of logarithms
\item Graphs of exponential functions and logarithms
\item The derivative of $a^x$ and the definition of $e$
\item Derivatives of Logarithms
\item Limits involving exponentials and logarithms
\item Exponential growth and decay
\end{itemize}

\section*{Week 9}
\begin{itemize}
\item \textbf{Chapter 8. The Integral}
\item Area under a Graph
\item When $f$ changes its sign
\item The Fundamental Theorem of Calculus
\item The summation notation
\item The indefinite integral
\item Properties of the Integral
\end{itemize}

\section*{Week 10}
\begin{itemize}
\item The definite integral as a function of its integration bounds
\item Method of substitution
\item \textbf{Chapter 9. Applications of the Integral.}
\item Areas between graphs
\end{itemize}

\section*{Week 11}
\begin{itemize}
\item Cavalieri's principle and volumes of solids
\item Three examples of volume computations of solids of revolution
\item Volumes by cylindrical shells
\item Distance from velocity
\end{itemize}

\section*{Week 12}
\begin{itemize}
\item The length of a curve
\item Velocity from acceleration
\item Work done by a force
\end{itemize}
\end{document}
