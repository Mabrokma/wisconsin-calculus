\documentclass{amsbook}
\input ../preamble.tex
\input ../free-macros.tex
\begin{document}
\begin{center}
  \bfseries\color{badgerred}\huge%
  When not to use l'Hopital's rule
\end{center}
\bigskip

l'Hopital's rule is very popular because it promises an automatic way of
computing limits of the form
\[
\lim_{x\to a } \frac{f(x)} {g(x)} = \text{\large``} \frac{0} {0}\text{\large''}.
\]
Instead of figuring out a trick that transforms the fraction into some other
expression whose limit we can compute, l'Hopital tells us just to differentiate
both $f(x)$ and $g(x)$ and then try again (repeat as necessary).  There are many
examples where l'Hopital's rule is the simplest way to an answer.  But there are
also many examples where l'Hopital makes you work much harder than necessary,
and even a few where l'Hopital just doesn't get you to the answer.

\section{Example -- Compute \(\displaystyle\lim_{x\to0}\frac{\sin x^3}{\sin^3x}\)}

The limit 
\begin{equation}
  \lim_{x\to0} \frac{\sin x^3}{(\sin x)^3}
  \label{eq:sinx3-oversin3x}
\end{equation}
is of the form $\frac 00$ so l'Hopital's rule applies.  
Recall that this rule says that if 
\[
  \lim_{x\to0} f(x) = 0 \text{ and }     \lim_{x\to0} g(x) = 0
\]
then
\[
  \lim_{x\to0} \frac{f(x)}{g(x)} = \lim_{x\to0} \frac{f'(x)}{g'(x)},
\]
provided the second limit exists.  

Let's see how useful l'Hopital is in computing $\lim_{x\to0} (\sin x^3)/(\sin
x)^3$.


\subsection*{Without l'Hopital's rule}

\begin{equation}
  \lim_{x\to0} \frac{\sin x^3}{\sin^3 x} = 
  \lim_{x\to0} \frac{\sin x^3}{x^3} \frac{x^3}{\sin^3 x}=
  \lim_{x\to0} \frac{\sin x^3}{x^3} \left(\frac{x}{\sin x}\right)^3=
  1\cdot 1^3 = 1. \qquad \text{\dfnt Done!}
  \label{eq:withoutlHopital}
\end{equation}

\subsection*{With l'Hopital's rule}
\begin{align*}
  \lim_{x\to0} \frac{\sin x^3}{\sin^3 x} &= \text{\large``}\frac00 \text{\large''}
  &\text{\dfnt so we can use l'Hopital}\\
  &= \lim_{x\to0} \frac{3x^2\; \cos x^3}{3\sin^2 x \; \cos x} &\text{\dfnt differentiated
  top \& bottom}\\
  & =  \text{\large``}\frac00{} \text{\large''}
  &\text{\dfnt so we can use l'Hopital again}\\
  &= \lim_{x\to0} \frac{6x \cos x^3 - 9x^4\sin x^3}
  {6\sin x\; \cos^2 x - 3\sin^3 x} 
  &\text{\dfnt differentiated top \& bottom}\\
  & = \text{\large``}\frac00\text{\large''} &\text{\dfnt so we can use l'Hopital again}\\
  &= \lim_{x\to0} \frac{6 \cos x^3 - 18x^3 \sin x^3 - 36x^3\sin x^3 - 27x^6 \cos x^3}
  {6 \cos^3 x -12\sin^2 x\;\cos x - 9\sin^2 x\;\cos x} 
  &\text{\dfnt differentiated top \& bottom}\\
  &=\frac66\\
  &=1.&\text{\dfnt Done!}
\end{align*}

\section{Other examples}
Here are two limits that are all of the form $\frac{0} {0}$, and to which one
could in principle apply l'Hopital's rule.  For these particular limits
l'Hopital leads to severe headaches, while there are simple ways of finding the
limits.
\begin{align*}
  I&=\lim_{x\to0} \frac{\sin x^5} {\sin^5 x} \\
  J&=\lim_{x\to\infty} \frac{e^x} {e^x + e^{-x}}
\end{align*}
Both of these limits is equal to $1$.

\section{Solutions}
The limit $I$ is just like the one we did when we
computed~\eqref{eq:sinx3-oversin3x}.  Using l'Hopital you get a really long
computation, but you can also compute it as in~\eqref{eq:withoutlHopital}.

The limit $J$ just turns into itself after applying l'Hopital twice.  The
better way is to divide top and bottom by $e^x$ (the biggest term around).

\end{document}
